\section{Codomains Introduction}
\label{sec:codomains-intro}

% the problem setup

When we build an application that requires the participation of mutually
distrustful parties, our choices are: 
%
\begin{enumerate}
    \item Run the entire application in a secure execution environment, such as
        a hardware enclave, or via translation to an MPC program.

    \item Distribute the application;  each party locally executes
        their sensitive piece of the computation.

    \item Hybrid, ``mix-mode", schemes whereby each party locally executes
        their sensitive piece of the computation, and jointly invoke a secure
        enclave or MPC scheme for computations that combine the parties'
        sensitive data.
\end{enumerate}


Our focus is designing and implementing operating system abstractions 
that enable developers to write such hybrid applications as if they were
monolithic, local applications.
%
This motivation is not conceptually different from that of 
remote procedure calls (RPCs):  with RPCs, developers code their applications
using the familiar procedure call, but the calling process and the
process containing the procedure being called execute on different
hosts.
%
To the application proper, the fact that the call involves cross-host I/O
is mostly transparent.
%
In our setting, the system abstractions define party-local and multi-party
secure computations, and the underlying system manages moving the execution
between the computation domains, maintaining resource consistency across
domains, and restricting information flow among domains.



As with RPCs, prior work in this area endeavors to hide the details of the
resultant distributed system from the application developer.  
%
Projects such as privtrans and Glamdring use developer-provided source code
annotations and static analysis to assist the compiler in partitioning the
monolithic application into separate programs.
%
In contrast, Fabric, Wysteria, and secure goroutines expose partitioning as
first-class language constructs, whereby developers programmatically indicate
which statements execute in a specific domain, and what data is accessible to
that domain.
%
In these systems, a trusted language runtime enforces isolation between
domains.


Similar to prior work, we want to maintain the source-level abstraction of
a monolithic program, but allow for dynamic, at run-time, mode switches.
%
Morever, we want to present a language-neutral mechanism for switching modes.
%
If mode-switching is dynamic and language-neutral, then orthogonal, application
interposition techniques, may be layered onto the system to increase the
transparency of the multi-domain environment with respect to the application.
