\section{Proposed Gemini Evaluation}
\label{sec:gemini-eval}
% Short intro that covers the gist: correctness and performance concerns.
%
% each subsection a question.  One or two paragraphs that expound upon the
% question and point to an evaluation method/tool.


\subsection{Correctness}
%
I need to ensure that a process running on Gemini shows the correct behavior
(program correctness), the isolation of pinned data (security correctness), and
the accuracy of the firewall rules (again, security correctness).
%
Pragmatically, this amounts to writing unit tests.
%
Additionally, it may involve showing generality by, for instance, instrumenting
more than just OpenSSL (the macro-benchmarks can also help to show generality
at the application-level).
%
For demonstrating data isolation, you might have an external program that just
dumps and scans a process's memory for a given byte string.


% TODO: try to think of a few correctness conditions, and what the unit test
% would look like for each one.  You might also consider tests for different
% application models, such as a single-threaded vs multi-threaded, vs
% multi-process application.


% TODO:


\subsection{Performance}

\subsubsection{Micro-benchmarks}

\parhead{How precise is taint-tracking?}
% - how do you know that taint tracking is being performed correctly
%   (not overtaintint, not undertainting).
% - eval at the call graph level 
%   take eleos annoations as hints --
%
Taint-tracking implementations may be susceptible to over-tainting or
under-tainting.
%
We develop a profiling tool that logs the functions that operate on
tainted data, and from these logs construct the call-graph the target machine
evaluates.
%
We use this tool to collect the call-graphs for migrations that compute a
signature
%
We compare this call-graph to the call-graph induced by prior manual or
static-analysis-based partitioning schemes applied to OpenSSL\@.
%
We also evaluate whether the call graph is stable across migrations.


\parhead{What is the overhead of migration?}
% The idea is:
%   - understand the mechanics of migration and "profile" each of its steps.
%       - this might point out certain optimizations
%   - compare to alterntavies
%       - to understand tradeoffs/overheads compared to alternatives
%
Suppose a thread migrates to a target machine that has pinned a private key in
order to compute a signature with that key.
%
At the network level, how would a packet capture of the migration compare to
that of an alternative RPC-based implementation, such as keyless SSL\@?
%
In particular, how much traffic and how many round-trips does migration incur,
what is the purpose of each round-trip, and what is the corresponding ratio of
goodput to throughput?
%
Moreover, is each migration identical?


Closely related to the previous question, we seek to measure the latency of
migration and the related operations.
%
Direct overhead costs include network latency, while indirect costs include the
operations of checkpointing and restoring threads, as well as any TCP queuing
delays incurred by the source machine while its threads are paused, waiting for
the migration to return.


\parhead{How much time does the application spend on each machine?}
%
Since the use-cases involve an organization outsourcing an application to a
service provider, it would defeat the purpose if the application spent the
majority of the time on organization's machine.
%
Note that ``time" may not be the correct metric.


\parhead{What is the overhead of taint tracking and policy enforcement?}

We evalute the overhead of the monitor performing taint tracking on some common
set of benchmarks (e.g., SPEC), adjusting the amount of data that is tainted.


\subsubsection{Macro-benchmarks}

% nginx: apachebench
% bind, knot DNS, nds: dnsperf
% exim, postfix, qmail, sendmail: smtp-source
%
Give our use-cases of an HTTPS, DNS, and mail server, we stress test an
implementation of each server running in a standard Linux environment, and
compare request throughput and latency to the server running on top of Gemini.
%
Specifically we stress test the NGINX webserver using ApacheBench, the BIND DNS
server using \texttt{dnsperf}, and the postfix mail server using
\texttt{smtp-source}.


For each server, we benchmark several deployment scenarios.
%
In particular, does the deployment have only the organization pinning data (one
or more private keys), or does the service provider itself also pin data (such
as a spam or web application firewall rules).
%
In the former case, only the organization runs a process-monitor.


Furthermore, does the service support only one organization, or multiple?
%
For instance, the NGINX webserver can virtually host many sites, and thus,
under Gemini, could migrate to different organizations based on the request.
%
The application server might also support different server models, such as
multi-process or multi-threaded.
%
In a multi-process, multi-tenant configuration of NGINX, each worker process
is single-threaded and could migrate to a different organization
simultaneously, whereas in a multi-threaded configuration, only a single thread
could migrate away from the service provider any time.


Additionally, what is the granularity of the firewall rules each organization
installs on the service provider's Gemini instance?
%
Are there no firewall rules, and Gemini is used purely for migration?
%
Are the firewalls simply inspecting taint at the protocol-field level, or
performing deep packet inspection, and using an in-kernel
implementation of TLS to inspect encrypted packets. 



% TODO: How is each server's evaluation different from one another?
% That is, are you able to evaluate something on a mail server that you aren't
% able to evalute on a webserver?
