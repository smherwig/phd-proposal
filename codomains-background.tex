\section{Codomains Background}
\label{sec:codomains-background}

We focus on two use cases that involve multiple parties cooperatively running
an application where the input of at least one party is confidential:
(1) outsourcing TLS-based applications to a third party, and (2) federated
analysis over private data.


\parhead{Outsourcing TLS applications}
%
In this setup, a company uses a third party, such as a cloud host or cloud
service, to operate an application server, such as for web, DNS, or mail.
%
Operationally, the status quo is for the company to shares or otherwise entrust
the third party with the application's private keys.
%
This has significant implications on the trust model of the PKI and the web
writ large, as these third parties can arbitrarily impersonate any of their
customer organizations.
%
Our goal is to design system primitives that allow developers to write or
patch, and deployers to configure, such applications with the property that the
company retains exclusive custody of its private keys.


\parhead{Collaborative Data Processing}
%
In this setup,  the aggregate dataset consists of the confidential, local,
datasets of multiple parties.
% 
The parties themselves might be mutually distrustful, or may simply be
prevented by law from sharing their data.
%
An analyst wants to process the datasets to determine some non-confidential
result.
%
With federated databases, the result may be the cardinality of an intersection
across the datasets.
%
With federated learning, the result may be the global model parameters (such
as the weights of a deep neural network), derived from the aggregation of the
parameters from each party's locally-trained model.
%
Our goal is to enable the developer to express, in a
single program, the multi-party nature of the application, with the underlying
system managing the application's distribution across parties. 


While our system will present an interface for partitioning the application
across the parties, partitionioning alone does not prevent private data
leakage.
%
For instance, local results and parameters may, by themselves, reveal
information about the private data samples.
%
Thus, orthogonal to our system, the application may additionally require
methods for achieving differential privacy, such as by adding a carefully
chosen amount of noise to the local results.


\subsection{Threat Model}

We assume an honest but curious (HbC) threat model: each party faithfully
executes the program, but may analyze their portion of the execution to try to
recover the private data of other principals.
%
We assume that the parties trust the application; that is, the application does
not deviate abitrarily from its stated purpose, and does not actively try to
leak sensitve data.
%
We assume that the application is bug-free, and thus our model omits external
actors, such as a remote attacker.


An HbC model is not inherent, but is reasonable for our use cases,
given the mutually beneficial incentives among the parties, and further
simplifies the system design.
%
Consider, in contrast, a more adversarial model where the parties, or the
application itself, may abuse the interface for domain switching---as by
switching at an unexpected state, or switching to a malicious code page---for
the purpose of actively leaking sensitive data. 
%
In this model, the target domains must first, externally, validate the
execution state against an expected starting condition before commencing
execution.


An interesting design point related to the threat model is whether the
application is globally public, or whether some portions are private and thus
confidential to a party. 
%
For instance, in the scenario of a company outsourcing a web server to a CDN,
the CDN's web server may be proprietary, and the CDN operator in turn may be
unwilling to allow the server's execution to switch to a company's domain
for fear of leaking proprietary knowledge.
%
Naively, this scenario reduces to specifying that private code pages should be
pinned to the owning domain, and that execution may switch domains only at
non-private points in the application's execution.
%
The tricky aspect is that the data written by private code pages should,
likewise, be private, as sharing such data pages could reveal aspects of
the private code.
%
We consider this feature as future work, but briefly revisit the idea in
\S\ref{sec:codomains-challenges}.


Typical of other work in this area, we consider side-channels out of scope.


%Within the realm of secure, federated data processing, the need for differential
%privacy is virtually unavoidable, as local results and parameters may, by
%themselves, leak information about the underlying data samples.
%%
%Differential privacy (DP) is a property of randomized queries that take a
%database as input and return a statistical result, such as an aggregate.
%%
%The database is a collection of rows, with each row containing data
%from one individual.
%%
%Informally, queries are differentially private if arbitrary changes to a single
%individual’s row result in only statistically insignificant changes in the
%function’s output distribution. 
%%
%Thus, the presence or absence of any individual has a statistically negligible
%effect.
%%
%Practical solutions for achieving DP typically rely on adding a carefully
%chosen amount of noise to the result.
%%
%DP offers a provable bound on the amount of information that an adversary can
%learn about any individual, even with access to auxiliary information


% DP Definition
%
%Differential privacy (DP): by
%disallowing certain qeuries and carefully adding a chosen amount of noise to
%the result of others, it is possible to give a strong upper bound on how mcuh
%an adversary could learn about an invididual person's data.
%


% such as an aggregate statistical value, or to learn the parameters to
%a machine learning model trained across the datasets.


%%%% FEDERATED LEARNING
%
%Federated Learning (FL) trains an algorithm across
%multiple principals holding local, confidential, data samples, thereby allowing
%the principals to build a common machine learning model withouth
%sharing their data.
%%
%%
%As local parameters may, by themselves, leak information about the underlying
%datasamples, FL may further incorporate DP or secure aggregation.


%DStress is a system that can efficiently perform computations on graphs taht
%contain confidential data. Dstress assumes that the graph is physically
%distributed across many participants, and that each participaint only knows a
%small subgraph; it protects privacy by enforcing tight, provable limits on how
%much each participant can learn about the rest of the graph.
%
%The motivating usefaul is measuring systemic risk in financial networks. The
%requried information is extrmeemly sensisive because it directly reflects the
%business strategy of each bank.
%
%It is well known taht many interesting thnigs can be learned by collecting and
%analyzing large graphs. Tools assume that the user has a property Graphen G
%(that is, a graph that has some data asscoiated with its vertixes and/or edges)
%and wishes to compute com function F(G) over this graph and itsp roperites.
%
%However, there is another class of use cases where the graph G contians
%sensitvie information and is spread across multiple administratigve daomins. In
%this situation, each domain knows only a subset of the vertexes and edges, so
%it cannot compute F(G) on its own, but the domains may not be willing to share
%their data with each other because of privacy concerns.
%
%
%DStress support vertex programs; it addresses the first challenge with a
%special graph-computation runtime tha can execute vertex programs in a
%distributed fashion, using MPC and a varint of ElGamal encryption
%(homomorphic for transferring data between domains, and it addresses the
%second challenge by keeping intermediate results encrypted at all times, nad by
%offering differential privacy on the final result.
%
%Our approach is based on two key insights. Our first observation is that much
%of the enormous cost of the MPC-based strawman comes from the fact that the
%graph is itself confidential and therefore must be an input to the computation.
%We can get around this by fomrulating the funciton F as a vertex program – that
%is, as a sequence of comptuations at each vertex that are interlevaed with
%message exchagnes over the edges – and by executing it in a distributed
%fashion. The main challenge is to prevent information leackage thorugh
%intermediate results. In DStress, we asccomplish this with a combination of
%secret sharing, small MPC invocations for the computations at each vertex, and a
%special protocol for transferring shares without revelaing the topology of the
%graph. our second key insight is that we can use differential privacy to
%achieve output privacy.


