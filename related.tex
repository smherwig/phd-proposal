\section{Related Work}
\label{sec:related}
% You want to keep this to no more than 3.5 pages (closer to 3).


\subsection{Trusted Execution Environments}

Trusted execution environments (TEEs) provide hardware protections for running
trusted portions of code with guarantees of confidentiality and integrity.  
%
Applications can be guaranteed that code executed within the TEE was run
correctly and that any secrets generated during execution will remain safely
within it as well.

A wide range of TEEs are available today, with varying functionalities.
%
We focus on Intel's Software Guard Extensions (SGX) environment, and AMD's
Secure Encrypted Virtualization, the two TEE implementations for the x86\_64
architecture.


\parhead{Intel SGX Overview}
%
Intel's SGX provides a new mechanism for trusted
hardware and software as an extension to the x86 instruction set~\cite{sgx,
mckeen2013innovative}.  
%
A program called an \textit{enclave} runs at high
privilege in isolation on the processor in order to provide trusted code
execution, while an untrusted application can make calls into the enclave.
%
While these enclaves can be statically disassembled (so the code running in the
enclave is not private), once an enclave is running, its internal state is
opaque to any observer (even one with physical access), as are any secrets generated.  


Enclaves must be measured and signed by their creator and cannot run without
this signature, and the enclave state is checked against this measurement
before running.  
%
An enclave can also cryptographically \textit{attest} to its current state, in
order to prove that it correctly executed code \cite{sgx_provisioning,
anati2013innovative}.  
%
Another feature is the ability to cryptographically \textit{seal} data to be
used across multiple invocations of an enclave~\cite{anati2013innovative,
sgx_sealing}.  
%
SGX also provides such features as trusted time and monotonic counters
\cite{sgx-linux-sdk,sgx-trusted-time}.  
%
However, an enclave currently has no access to networking functionality itself,
so it must rely on the untrusted application for all network interactions.  
%
In fact, enclaves are prohibited from making any system calls, so these must be
proxied through the untrusted OS as well.

\parhead{AMD-SEV Overview}
%
% Threat Model
%%--------------------
AMD Memory Encryption Technology addresses two different classes of attacks:
system software level and physical access attacks.  
%
System software level attacks include attacks by the hypervisor or co-reseident
VMs to analyze the guest VM memory space.
%
Physical attacks include hot memory I/O attacks or cold boot attacks.


% Design
%-------------------------
AMD Memory Encryption Technology introdues an AES-128 encryption engine inside
the System on Chip (SoC) that transparently encrypts and decrypts the data
when it leaves or enters the SoC, respectively.
%
AMD proposes two main features based on Memory Encryption Technology: (1)
Secure Memory Encryption (SME) and (2) Secure Encryption Virtualization (SEV)\@.
%
Both SEV and SME are managed by the OS or hypervisor, and no application software
changes are needed.
%
AMD's Memory Encryption Technology does not provide memory integrity
protection.


% AMD secure coprocessor
%-------------------------
Encryption key management, such as generating, storing, and delivering the
keys, are carried out by the AMD secure processor and the encryption keys are kept
hidden from untrusted parts of the platform.
%
%The AMD secure processor utilizes a 32-bit ARM Cortex A5, and uses its memory
%an storage while executing a kernel that is signed by AMD\@.
%
The secure processor provides a set of APIs for provisioning and managing
the platform in the cloud.


% Secure Memory Encryption (SME), and 
% Transparent Secure memory Encryption (TSME)
%---------------------------------------
SEM is the security feature that addresses physical access atacks.
%
SEM is an x86 instruction set extension introduced by AMD for page-granular
memory encryption support using a single ephemeral key.
%
The AMD secure processor randomly generates the key and loads it into the
memory controller at boot time.
%
The OS is able to leverage the SME by setting a bit in the x86 page table (the
en\emph{C}rypted, or \emph{C-bit}).
%
When the C-bit is set, access to that memory page is directed to the AMD Memory
Encryption Engine.
%
In the SME design, all devices can access the encrypted pages through DMA\@.


%In the case of legacy OSes, a feature called Transparent Secure Memory
%Encryption (TSME) allows the C-bit to effectively be set in the BIOS for all
%memory pages.


% Secure Encrypted Virtualization 
%---------------------------------------
SEV extends SME to AMD-V, allowing individual VMs to run SEM using their own
secure keys.
%
SEV encrypts and protects the VM's memory space from the hypervisor and
co-resident VMs using a VM-specific encryption key.
%
When code and data arrive into the SoC, SEV tags all of the code and data
associated with the guest VM in the cache and limits access only to the tag's
owner VM.
%
%Under SEV, the ASID field in the page table is used as the key index that
%identifies which encryption key is used to encrypt and decrypt the memory
%traffic associated with that VM.
%
SEV-enabled VMs can control their own C-bit for memory pages they want
encrypted.


% Attestation and Provisioning
%------------------------------
%The AMD secure processor provides a set of APIs for provisioning and managing
%the platform in the cloud.
%
%The hypervisor's SEV driver can invoke these APIs.
%
%In the SEV architectures, a guest owner manages her guest secret and generates
%the policies for VM migration of debugging.
%%
%The Diffie-Hellman key exchange protocol is used between the guest owner and the
%AMD secure processor to open a channel between the guest owner and the AMD
%secure processor; the guest owner is able to authenticate the secure
%processor and exchange information to set up the protected VM.
%%
%The guest owner can set a policy where certain pages are private and others
%shared; if shared, the hypervisor can read it in plain text.
%
%
%SEV may be used in conjunction with SME.  Under this scenario, each SEV-enabled
%VM controls its own encryption via the C-bit and the host page tables control
%the encryption for shared memory.


\parhead{Attacks}
%
We must address the recent rise of side-channel attacks against SGX, including
the speculative execution attack Foreshadow~\cite{foreshadow,
weisse2018foreshadow}.  
%
This attack allows for the extraction of not only the entire SGX enclave's
memory contents but also the attestation and sealing keys.  
%
We note that this attack would break the security guarantees that we provide
with conclaves.
%
Intel has stated that SGX is explicitly designed to not deal with side-channel
attacks in its current state and leaves handling this up to enclave
developers~\cite{sgx-sidechannel, sgx-developers}.
%
Regardless, Intel has released both microcode patches and recommendations for
system level code that at the current time address Foreshadow and known related
attacks \cite{sgx-patch, canella2018systematic, weisse2018foreshadow}.  
%
There is also ongoing research to address both speculative execution as well as
other cache-based side-channel attacks on SGX and in general
\cite{yan2018invisispec, oleksenko2018varys, canella2018systematic, shih2017t}.

% TODO: add paragraph or two about attacks on SEV.


% Each solution operates at a differnet level in the stack, more or less.
% Handle TEE and non-TEE based solutions together.

\subsection{Application-Level Solutions}
% here, you can talk about program partitioning, both those that are aware of
% TEEs, and those that came before.

\parhead{HTTP Solutions}
% TODO: others (NoCDN, SINE, OCDN, S-HTTP, Data Staging on Untrusted
% Surrogates)
%
Several systems have proposed that the origin server digitally
sign their data~\cite{cdn-on-demand,stickler} or embed cryptographic hashes
directly into
HTML~\cite{w3c-subresource-integrity,w3c-content-security-policy}, which
clients can then verify.
%
Such approaches ensure provenance, freshness, and integrity of web assets
served by a proxy---without requiring the proxy to store the origin server's
private key.
%
However, they do not provide for confidentiality, nor do they allow for CDN
services such as media transcoding and web application firewalls.
%
Moreover, they place the origin on the critical path, thereby increasing
latency and making them more susceptible to attack.


\parhead{TLS Solutions}
%
Other work allows the organization to retain ownership of its private keys by
changing the server-side implementation of TLS.
%
SSL Splitting~\cite{ssl-splitting} leverages the fact that a TLS stream
comprises data records and authentication records (MACs), and develops a new
protocol in which the organization sends the authentication records and the
provider merges them with the data records to form the complete TLS stream.
%
In essence, this implements the above HTTP solutions in TLS, and thus suffers
from the same limitations of requiring the origin server to be on the fast
path.

Cloudflare's Keyless SSL~\cite{keyless-ssl} takes advantage of the fact that
TLS only uses the website's private key in a single step of the TLS handshake.
%
Like SSL Splitting, Keyless SSL keeps the master private key off of, and unknown
to, the proxy, but unlike SSL Splitting, Keyless SSL does not provide for
content provider endorsement of the content the proxy serves.  
%
Neither SSL splitting nor Keyless SSL provides for the protection of the
session keys from the provider.


\subsection{Cryptographic Solutions}
% In addition to homomorphic encryption, also need to deal with verifiable
% computation.
%
One seemingly straightforward approach to solving this problem would seem to be
fully homomorphic encryption (FHE) or functional
encryption~\cite{gentry2009fully, gentry2010computing, garg2016candidate}.
%
FHE allows one to perform arbitrary computations on \textit{encrypted} data,
without knowing any of the keys.  
%

However, even current state-of-the-art homomorphic encryption is much too slow
for the performance that is required for our use-cases, and additionally would
violate our goal to support legacy applications.


Various approaches~\cite{desmoulins2018pattern,
sherry2015blindbox, canard2017blindids,lan2016embark} apply searchable
encryption schemes to achieve functionality like deep packet inspection (DPI)
while still maintaining the privacy of data.
%
In general, these approaches require changes of some sort to the endpoint(s),
suffer from performance overheads, and do not achieve the rich and varied
features offered by providers.


\subsection{Execution Environment Solutions}
% se, here, you talk both about IFC as well as Running legacy applications in
% SGX.

\parhead{Running Legacy Applications on SGX}
%
Various works use SGX as a mechanism for achieving shielded execution of
unmodified legacy applications.
%
These works generally differ in how much of the application's code runs
within the enclave.
%% , and the complexity of the interface between the enclaved
%% and non-enclaved portions.


At one extreme, TaLoS~\cite{talos} simply ports the LibreSSL library to SGX so
that the application terminates TLS connections in an enclave; the rest of
the application remains outside the enclave, unchanged.
%
This approach protects the private keys and session
keys, but does not address our goals of multitenancy or WAFs.


At the other extreme, SCONE~\cite{scone} moves the entire C library into the enclave.
%
%% SCONE~\cite{scone} instead moves the entire C library into the enclave, and
Haven~\cite{haven} and Graphene~\cite{graphene} carry this approach further by
implementing kernel functionality in an enclave by means of a library operating
system (libOS).
libOSes refactor a traditional OS kernel into a user-land library that loads a
program.
%
The program's C library is modified to redirect system calls to the libOS, which
in turn either services the calls internally or calls into the untrusted OS
when the host's resources are needed.
%
Aurora~\cite{liang2018aurora} extends the libOS from the SGX enclave to System
Management Mode (SMM) by running device drivers in SMM memory.


CDN applications involve multiple processes, and of these works, only Graphene
supports forking and executing new processes within enclaves.
%
However, Graphene's support for shared state among multiple enclaves, such as a
read-write file system and shared memory, is limited.
%
We discuss these limitations in \S\ref{sec:design} and our extensions to
Graphene in \S\ref{sec:implementation}.


Other work~\cite{beekman2016improving} provides
%% We also note projects such as~\cite{beekman2016improving} which present
frameworks for developing \emph{new} software that takes advantage of SGX,
whereas our interest is in supporting \emph{legacy} applications.

% Aurora ~\cite{liang2018aurora}  takes a
% slightly different approach, but is still not well-suited to the multi-tenant
% case, in particular for isolation of the filesystems, memory, and time.

% Beekman~\cite{beekman2016improving} seeks to build a way for Internet service
% providers to host secure services that protect the confidentiality and
% integrity of a user's data using SGX.  While our goals are similar in nature,
% the approaches differ in system-level that provides the services and isolation.
% Beekman's secure service mechanism operates at the application level and
% requires application changes.  Our mechanism is at the OS-level and supports
% legacy applications.

%but I think the gist is that aurora installs it's own little operating system
%into System Management Mode, which is super-privileged and typically only for
%the BIOS's operation.

%I think the difference is likely that that design doesn't quite address
%multiple users/customers; in particular, isolating the the filesystems,
%memory, time, etc. like we do;

%it's also not clear to me that SMM has the same properties as the EPC
%that is, the memory might be readable by someone with physical access
%I'm not very familiar with SMM



% Prior work on SGX libOSes~\cite{graphene,scone,haven} make it possible
% to run legacy applications within an SGX enclave, but all of them
% either lack multiprocess support completely, or only support multiple
% processes in a restricted environment.

%
% Moved to prior work
%% libOSes refactor a traditional OS kernel into a user-land library that loads
%% the program.
%% %
%% The program's C library is modified to redirect system calls to the libOS; the
%% libOS either services these calls internally, or calls into the host kernel
%% when host resources are needed.

\parhead{Information Flow Control}
%
Decentralized information flow control (DIFC) enhanced operating systems allow
dynamic labeling of OS abstrations, such as processes and files, and enforce
access control based on Denning's lattic model for information flow
security~\cite{}.
%
The tags represent different rights (capabilities) with respect to the data
that the process is allowed to read or write.
%
Using tags, these systems can enforce the classical secrecy and integrity policies of
``no read up, no write down", or an integrity policy of ``no read down, no
write up", between the objects in the system.
%
Additional policies, such as \emph{export protection} are possible, in which a
process cannot have both an uncontrolled channel (e.g., a socket) open and
access to private data that it cannot declassify, as well as
policies for stringent system-wide \emph{read-protection} and \emph{integrity
protection}.
%
These process-level information flow models are coarse grained and cannot track
sensitive information \emph{within} untrusted applications.


\parhead{Fine-Grained IFC}
%
Fine-grained information flow tracking (IFT) is a technique whereby execution
runtime tags memory and registers containing sensitive data with labels (also
called taint or taint markings) and propagates these labels in
acccordance with the computation.
%
Instrumentation is transparent to the application process, as the application
observes the same addresses and same values as it would in an uninstrumented
execution.
%
IFT traditionally uses a source and sink model whereby labels are
assigned at sources and checked at sinks; the labels themselves are opaque,
and are interpreted by application-specific policy.
%
Most work in this area has been specifically motivated by detection of control
flow hijacking attacks, such as buffer overflows, and involves tainting of
program inputs, and trapping of control transfers to tainted target addresses.
%
However, IFT is also used to enforce mandatory access control on data and
derived data, and thus a restriction on the flow of information between
the system's objects (e.g., users, processes, files).


Fine-grained IFT can either be applied at the process- or system-level.
%
A standard approach for process-level information flow tracking is to run
the process in an emulator, such as Valgrind, Intel Pin, or DynamoRIO; the
emulator serves as a reference monitor, and dynamically tracks data flows.
%
In contrast, full-system IFT involves running the guest system in a hardware
emulator (e.g., QEMU) that has been augmented with machine instruction analysis
and taint tracking capabilities.
%
As an optimization, and for security isolation, a hypervisor may dynamically
switch execution between emulated execution when processing tainted data, and
native virtualized execution when processing untainted data.
%
Full-system IFT systems also propagate taint to the filesystem (as by storing
taint in a file's extended attributes) and across the network (as by placing
labels in a packet's IP options).


\parhead{Uses with Secure Hardware}
% TODO: also talk about compiler-based techniques (e.g., Moat)
%
Ryoan presents a request-response execution model that allows mutually
distrustful parties to process sensitive data in a distributed fashion on
untrusted infrastructure, without any party leaking secret data.
%
Ryoan runs each party's processing modules in a trusted userspace sandbox,
which in turn is hosted in an SGX enclave.
%
Each module tags its output, essentially performing taint tracking at
enclave-level granularity, where the taint indicates which encalves have seen
secret data.
%
When operating on tainted data, Ryoan ensures that the module cannot leak data
by controlling explicit I/O channels, obfuscating network traffic, forbidding
use of the fileystem, and ensuring that input is only processed once.

