\section{Related Work}
\label{sec:related}
% background work that comprises your reading list
% 
% backgroud
% prior


Here, we review relevant background and prior work in terms of how they have
achieved the goals outlined in \S\ref{sec:goals}.
%
There have been a variety of approaches that achieve a subset of our goals, but
to the best of our knowledge, we are the first to achieve them all.
%
See Table~\ref{tbl:prior} for a comparison.


\subsection{TEE-less Solutions}

%% First, we describe prior software-based approaches, ultimately showing that,
%% currently, trusted hardware appears necessary to achieve our desired
%% functionality, performance, and support for legacy applications.

\parhead{HTTP Solutions}
% TODO: others (NoCDN, SINE, OCDN, S-HTTP, Data Staging on Untrusted
% Surrogates)
%
Several systems have proposed that the origin server digitally
sign their data~\cite{cdn-on-demand,stickler} or embed cryptographic hashes
directly into
HTML~\cite{w3c-subresource-integrity,w3c-content-security-policy}, which
clients can then verify.
%
Such approaches ensure provenance, freshness, and integrity of web assets
served by a proxy---without requiring the proxy to store the origin server's
private key.
%
However, they do not provide for confidentiality, nor do they allow for CDN
services such as media transcoding and web application firewalls.
%
Moreover, they place the origin on the critical path, thereby increasing
latency and making them more susceptible to attack.



%% CDN on Demand~\cite{cdn-on-demand} and Stickler~\cite{stickler} seek to
%% guarantee the provenance, freshness, and integrity of web assets served by a
%% proxy by having the origin sign the assets; the client then downloads a
%% trusted JavaScript loader that verifies the asset signatures.
%% %
%% The W3C Subresource Integrity~\cite{w3c-subresource-integrity} and Content
%% Security Policy~\cite{w3c-content-security-policy} recommendations are
%% renditions on this idea whereby the origin embeds the cryptographic hash of a
%% resource served by a third-party as a new tag attribute for the
%% \texttt{link} and \texttt{script} HTML elements, as well as sends a new HTTP
%% header to mandate the appearance of these tags for specific resources.
%% 
%% 
%% These solutions allow the TLS private key to remain on the origin server and
%% off the proxies, but place the origin on the critical path.
%% %
%% Furthermore, these solutions do not provide for the confidentiality of the
%% resources, and do not allow for CDN services such as media transcoding and
%% application firewalls.


\parhead{TLS Solutions}
%
Other work allows origin servers to retain ownership of their private keys by
changing the server-side implementation of TLS.
%
SSL Splitting~\cite{ssl-splitting} leverages the fact that a TLS stream
comprises data records and authentication records (MACs), and develops a new
protocol in which the origin sends the authentication records and the proxy
merges them with the data records to form the complete TLS stream.
%
In essence, this implements the above HTTP solutions in TLS, and thus suffers
from the same limitations of requiring the origin server to be on the fast
path.


%% A number of works seek to have the origin retain ownership of the private
%% key by changing the server-side implementation of TLS.
%% %
%% SSL Splitting~\cite{ssl-splitting} observes that the TLS stream is separable
%% into data records and authentication records, and develops a new protocol
%% between origin and proxy whereby the origin sends the authentication
%% records, and the proxy merges these with the data records to form the
%% complete TLS stream.
%% %
%% This protocol guarantees the provenance, integrity, and freshness of the web
%% resources, but involves the origin server in the entire response.
%
%The key insight of SSL splitting is that a stream of SSL records is separable into
%a data component and an authenticator component; each component uses a
%different key (encryption key and MAC key, respectively) derived independently
%from the master secret.
%
%In SSL Splitting, the origin server sends the SSL record authenticators, and
%the proxy merges them with a stream of message payloads (data records)
%retrieved from the proxy's cache; the merged stream is indistinguishable
%from a normal SSL connection between the client and the origin.
%
%
%Since the origin participates in every TLS session, SSL-splitting reduces only
%the bandwidth-load, and not the CPU load, of the origin server, and does not
%improve the redundancy of the site, as the origin must always be available
%to authenticate data.
%
%SSL splitting lacks end-to-end confidentiality; while the origin server retains
%the MAC key established during the SSL handshake, the origin must expose the session
%encryption key to the proxy.
%


%Rather than split TLS at the record level, 
Cloudflare's Keyless SSL~\cite{keyless-ssl} takes advantage of the fact that
TLS only uses the website's private key in a single step of the TLS handshake.
%
%Cloudflare thus allows the handshake to be split geographically, with most of the
%handshake happening at the edge server, and the private key operations proxied 
%to the remote server.
%
%If the customer operates the remote server on their own premises, then 
%the customer has exclusive access to the private key.
%
%The implementation requires changes to both OpenSSL and NGINX to proxy the
%private key operation in a non-blocking fashion.
%
Like SSL Splitting, Keyless SSL keeps the master private key off of, and unknown
to, the proxy, but unlike SSL Splitting, Keyless SSL does not provide for
content provider endorsement of the content the proxy serves.  
%
Neither SSL splitting nor Keyless SSL provides for the protection of the
session keys from the CDN provider.


Another line of work modifies TLS to allow for the interception of traffic by
middleboxes~\cite{naylor2015multi,naylor2017and,leematls}.
%
This is contrary to our desire to support legacy applications; it is not clear
how these solutions would be integrated with tools such as WAFs.
%
%% These solutions also largely seek to protect private and/or session keys and do
%% not discuss how they would be integrated with tools such as WAFs.

%mcTLS (SIGCOMM 2015)
%This one doesn’t seem to help us much
%Controls selective access to data/certain parts of the connection
%All computation/inspection would be done on plaintext
%Changes both endpoints?
%Concept of “context keys”
%
%mbTLS
%Looks like you need to tweak client slightly?? (but otherwise no endpoint changes)
%Proxies all connections through SGX
%Per hop session keys, stored in SGX
%Keys protected from middleboxes
%Each endpoint provisions its own middleboxes
%Would have to run all inspection in SGX on plaintext
%Claims with large buffers ~7Gbps throughput
%Smaller ones ~2Gbps

%\subsection{PKI Solutions}
%\paragraph{PKI Solutions}
%% \parhead{PKI Solutions}
%% %
%% Liang et al.~\cite{when-https-meets-cdn} approach the issue of composing
%% HTTPS with CDNs as a problem of specifying delegation.  
%% %
%% They propose a DANE-based~\cite{DANE} solution in which customers express
%% delegation by adding DNS resource records to their zone file for both their
%% certificate and the CDN's.
%% %
%% Unfortunately, the proposal requires changes to client certificate validation,
%% which breaks legacy support.
%% %
%% Also, while the proposal elegantly removes the need to share keys, it fails to
%% reduce the man-in-the-middle threats of a CDN, as the CDN can still alter
%% content and otherwise impersonate the customer in arbitrary ways.  


\parhead{Cryptographic Solutions}
%~\cite{blindbox},~\cite{homomorphic}
One seemingly straightforward approach to solving this problem would seem to be
fully homomorphic encryption (FHE) or functional
encryption~\cite{gentry2009fully, gentry2010computing, garg2016candidate}.
%
FHE allows one to perform arbitrary computations on \textit{encrypted} data,
without knowing any of the keys.  
%
However, even current state-of-the-art homomorphic encryption is much too slow
for the performance that is required of a CDN and additionally would violate
our goal to support legacy applications.


Various approaches~\cite{desmoulins2018pattern,
sherry2015blindbox, canard2017blindids,lan2016embark} apply searchable
encryption schemes to achieve functionality like deep packet inspection (DPI)
while still maintaining the privacy of data.
%
In general, these approaches require changes of some sort to the endpoint(s),
suffer from performance overheads, and do not achieve the rich and varied CDN
features we require.


%BlindBox (SIGCOMM 2015)
%Requires endpoint changes
%Slow connection setup for many rules
%Memory space??
%Running regex requires selective decrypt and can only be done on plaintext ? “probable decryption”
%Allows us to do keyword matching over encrypted data
%Per connection setup/rule encryption/circuit generation
%Symmetric crypto [some word I can’t read, need to figure that out]
%
%SGX-Box (APNet 2017)
%Run all filtering/inspection in SGX
%Provision session keys to enclave via out of band channel
%Is it efficient to run everything in SGX???
%5k patterns = ~3.5Gbps throughput
%
%BlindIDS (AsiaCCS 2017)
%Encrypt decryption pattern/rules only once overall
%“Decryptable searchable encryption”
%Preserves privacy of detection rules (do we care about this?)
%Shifts much of the overhead to detection phase on SP ? performance doesn’t seem that great
%Good for post-intrusion detection
%Load time better than BlindBox but still fairly slow….
%Pairing based
% only honest but curious middleboxes
%
%Embark (NSDI 2016)
%Outsourcing middleboxes to clouds
%Improvements come from architecture/setting changes? (which doesn’t really work for us)
%
%Pattern Matching on Encrypted Data (ePrint 2017)
%Pairing based
%No actual performance numbers?
%Large public keys (linear in size of plaintext) and large ciphertexts (same)
%Theory result???
%Can do regex+keyword search
%Compares itself to BlindBox but doesn’t give any actual implementation or concrete performance numbers
%

\subsection{Intel SGX (and Other TEEs)}
\label{sec:sgxbackground}
%\clg{I just dumped a big block of text in here, will edit}
Trusted execution environments (TEEs) provide hardware protections for running
small trusted portions of code with guarantees of confidentiality and
integrity.  Applications can be guaranteed that code executed within the
TEE was run correctly and that any secrets generated during execution
will remain safely within it as well.

A wide range of TEEs are available today, with varying functionalities.
We focus on Intel's Software Guard Extensions (SGX) environment,
but note that any TEE with similar functionality discussed here and
\S\ref{sec:threatmodel}
% in the threat model 
% (such as ARM TrustZone~\cite{trustzone}) 
would also be usable.

%The functionality that we explain in "SGX Overview": isolation, trusted code
%execution, calls into/out of the enclave, attestation, cryptographic "sealing"
%of the data, and trusted time/monotonic counters.

%\subsubsection{SGX Overview}
%\paragraph{SGX Overview}
\parhead{SGX Overview}
% TODO: brief overview of intel SGX.
%\myparagraph{Intel's SGX} Intel's Software Guard Extensions (SGX) allow for
%the creation of \textit{enclaves}, a trusted execution environment\cite{sgx,
%mckeen2013innovative}.   These enclaves can be statically disassembled, but
%are opaque while running.  Enclaves can also \textit{attest} to their current
%state, proving correct execution\cite{sgx_provisioning, anati2013innovative}.
%SGX also provides such features as trusted time, monotonic counters, etc.
%\cite{sgx_sdk_guide}.
%SGX can cryptographically \textit{seal} data to be used across multiple
%invocations -- corresponding to the encrypted state used in our protocol.
%Data can either be sealed to the owner of the enclave or just to the specific
%enclave instance~\cite{anati2013innovative, sgx_sealing}.  Production enclaves
%can only be produced by obtaining a production license and signing key from
%Intel, which whitelists licensee's key~\cite{sgx_production,
%sgx_production_faq}.
Intel's SGX provides a new mechanism for trusted
hardware and software as an extension to the x86 instruction set~\cite{sgx,
mckeen2013innovative}.  A program called an \textit{enclave} runs at high
privilege in isolation on the processor in order to provide trusted code
execution, while an untrusted application can make calls into the enclave.
While these enclaves can be statically disassembled (so the code running in the
enclave is not private), once an enclave is running, its internal state is
opaque to any observer (even one with physical access), as are any secrets generated.  


%% Enclave memory resides in a special region of system memory called the Enclave
%% Page Cache (EPC).  
%% %
%% Code and data from multiple enclaves can reside within the EPC, but
%% each EPC page is owned by only a single enclave and this owner is the only one
%% allowed to access the page.
%% %
%% An on-chip Memory Encryption Engine (MEE)~\cite{sgx3} encrypts and integrity
%% protects the EPC; the MEE decrypts enclave memory when the memory is brought
%% into the CPU's cache.
%% %
%% In current SGX-capable systems, the EPC is 128 MiB, of which 93 MiB is usuable
%% (the rest is used to store metadata for integrity protections).


Enclaves must be measured and signed by their creator and cannot run without
this signature, and the enclave state is checked against this measurement
before running.  An enclave can also cryptographically \textit{attest} to its
current state, in order to prove that it correctly executed code
\cite{sgx_provisioning, anati2013innovative}.  Another feature is the ability
to cryptographically \textit{seal} data to be used across multiple invocations
of an enclave~\cite{anati2013innovative, sgx_sealing}.  SGX also provides such
features as trusted time and monotonic counters
\cite{sgx-linux-sdk,sgx-trusted-time}.  However, an enclave
currently has no access to networking functionality itself, so it must rely on
the untrusted application for all network interactions.  In fact, enclaves are 
prohibited from making any system calls, so these must be proxied through the
untrusted OS as well.

%An important feature of SGX is its ability to cryptographically \textit{seal}
%data to be used across multiple invocations of an enclave.  Data can either be
%sealed to the owner/creator of the enclave so that it is readable by other
%enclaves from the same vendor or just to the specific enclave itself
%\cite{anati2013innovative, sgx_sealing}. A specific seal key is derived from
%the master seal key that is fused onto the chip in
%manufacturing.\christina{Check on this to make sure I have it all right}

%Data within the enclave is not accessible by any external application unless
%explicitly written out.  Enclaves must be measured and signed by their creator
%and cannot run without this signature.  Enclaves can run in two main modes:
%DEBUG and RELEASE.  RELEASE mode provides full production-quality hardware
%protection for the enclave and keeps the enclave execution protected, while
%DEBUG mode allows the enclave to be examined and debugged
%\cite{sgx_production, sgx_enclave_modes}.  Production-quality enclaves can
%only be produced by obtaining a production license and signing key from Intel,
%which then places the licensee's key on a whitelist~\cite{sgx_production,
%sgx_production_faq}.


\parhead{Running Legacy Applications on SGX}
%
Various works use SGX as a mechanism for achieving shielded execution of
unmodified legacy applications.
%
These works generally differ in how much of the application's code runs
within the enclave.
%% , and the complexity of the interface between the enclaved
%% and non-enclaved portions.


At one extreme, TaLoS~\cite{talos} simply ports the LibreSSL library to SGX so
that the application terminates TLS connections in an enclave; the rest of
the application remains outside the enclave, unchanged.
%
This approach protects the private keys and session
keys, but does not address our goals of multitenancy or WAFs.


At the other extreme, SCONE~\cite{scone} moves the entire C library into the enclave.
%
%% SCONE~\cite{scone} instead moves the entire C library into the enclave, and
Haven~\cite{haven} and Graphene~\cite{graphene} carry this approach further by
implementing kernel functionality in an enclave by means of a library operating
system (libOS).
libOSes refactor a traditional OS kernel into a user-land library that loads a
program.
%
The program's C library is modified to redirect system calls to the libOS, which
in turn either services the calls internally or calls into the untrusted OS
when the host's resources are needed.
%
Aurora~\cite{liang2018aurora} extends the libOS from the SGX enclave to System
Management Mode (SMM) by running device drivers in SMM memory.


CDN applications involve multiple processes, and of these works, only Graphene
supports forking and executing new processes within enclaves.
%
However, Graphene's support for shared state among multiple enclaves, such as a
read-write file system and shared memory, is limited.
%
We discuss these limitations in \S\ref{sec:design} and our extensions to
Graphene in \S\ref{sec:implementation}.


Other work~\cite{beekman2016improving} provides
%% We also note projects such as~\cite{beekman2016improving} which present
frameworks for developing \emph{new} software that takes advantage of SGX,
whereas our interest is in supporting \emph{legacy} applications.

% Aurora ~\cite{liang2018aurora}  takes a
% slightly different approach, but is still not well-suited to the multi-tenant
% case, in particular for isolation of the filesystems, memory, and time.

% Beekman~\cite{beekman2016improving} seeks to build a way for Internet service
% providers to host secure services that protect the confidentiality and
% integrity of a user's data using SGX.  While our goals are similar in nature,
% the approaches differ in system-level that provides the services and isolation.
% Beekman's secure service mechanism operates at the application level and
% requires application changes.  Our mechanism is at the OS-level and supports
% legacy applications.

%but I think the gist is that aurora installs it's own little operating system
%into System Management Mode, which is super-privileged and typically only for
%the BIOS's operation.

%I think the difference is likely that that design doesn't quite address
%multiple users/customers; in particular, isolating the the filesystems,
%memory, time, etc. like we do;

%it's also not clear to me that SMM has the same properties as the EPC
%that is, the memory might be readable by someone with physical access
%I'm not very familiar with SMM


\parhead{TEEs and Middleboxes}
%
A recent series of works have explored securing middleboxes by using TEEs, to
provide DPI and intrusion
detection~\cite{han2017sgx,DBLP:journals/corr/abs-1802-00508}, as well as network function
virtualization~\cite{poddar2018safebricks,trach2018shieldbox,goltzsche2018endbox,DBLP:journals/corr/DuanYW17,bhardwaj2018spx}.
%
None of these systems handle the complete range of functionality required by CDNs, nor do they support
multitenancy, to the best of our knowledge.

% One such set of systems tackles the problems of DPI and intrusion detection
% (IDS)~\cite{han2017sgx, DBLP:journals/corr/abs-1802-00508}.  These are
% complementary to our work, as CDNs do typically utilize this functionality, but
% much more limited than the generic CDN functionality achieved by \projname.


% Another class of work deals with deploying general network functions (NFs) on
% middleboxes using SGX.  SafeBricks~\cite{poddar2018safebricks} works to enforce
% least privilege across NFs but cannot support any network functions that rely
% on operations prohibited within SGX, such as system calls and timestamps,
%    unlike \projname.  ShieldBox~\cite{trach2018shieldbox} also seeks to deploy
%    NFs within a TEE and leverages SCONE to allow for the handling of system
%    calls.  ENDBOX~\cite{goltzsche2018endbox} is designed to allow for a
%    distributed deployment of middlebox functions on client machines.  This is
%    contrary to our goal of supporting legacy applications and orthogonal to our
%    goal of building a more secure cloud-based CDN.
%    LightBox~\cite{DBLP:journals/corr/DuanYW17} allows for generic NF deployment
%    but focuses on protecting metadata information for encrypted traffic.  We
%    observe that none of these systems handle the complete range of
%    functionality required and note that many of these approaches are unable to
%    handle the multi-tenant setting that we support.
% 
% SPX~\cite{bhardwaj2018spx} also works to deploy generic functions at edge nodes
% but at a protocol level.
% %but does this by providing a framework for securely extending protocols to
% %handle secure edge computing needs.
% This provides a more generic set of supported communication protocols than our
% work but would not handle the other CDN system design constraints.

The most relevant works combining TEEs and middleboxes are
Harpocrates~\cite{ahmed2018harpocrates} and STYX~\cite{wei2017styx}.
%
Harpocrates builds basic CDN functionality using a TEE and alludes to
performing Keyless SSL-like functionality using trusted hardware but does not
provide any details.
%
In addition, Harpocrates does not seek to protect any derived key material and
instead focuses solely on protecting the long term private key.
%
STYX improves Keyless SSL by protecting private and session keys, but does not
address secure WAFs or other CDN-type functionality.


%% , as the authors feel that in order to do its job, a CDN must see
%% all requests and responses in plaintext.
%% %
%% STYX seeks to improve Keyless SSL by
%% building a hierarchical key distribution scheme to protect keys provisioned on
%% CDN nodes.
%% %
%% STYX deals only with the problem of improving Keyless SSL and
%% protecting private and session keys.
%% %
%% It does not address secure WAFs or other
%% CDN-type functionality.


%ShieldBox: Secure Middleboxes using Shielded Execution~\cite{trach2018shieldbox}\\
%ENDBOX: Scalable Middlebox Functions Using Client-Side Trusted Execution~\cite{goltzsche2018endbox}\\
%Harpocrates: Giving Out Your Secrets and Keeping Them Too~\cite{ahmed2018harpocrates}\\
%SPX: Preserving End-to-End Security for Edge Computing~\cite{bhardwaj2018spx}\\
%LightBox: Full-stack Protected Stateful Middlebox at Lightning Speed~\cite{DBLP:journals/corr/DuanYW17}\\
%SafeBricks: Shielding Network Functions in the Cloud~\cite{poddar2018safebricks}\\
%Snort Intrusion Detection System with Intel Software Guard Extension (Intel SGX)~\cite{DBLP:journals/corr/abs-1802-00508}
%STYX: a trusted and accelerated hierarchical SSL key management and distribution system for cloud based CDN application~\cite{wei2017styx}



%\subsubsection{Side-Channel Attacks}
%\paragraph{Side-Channel Attacks}
\parhead{Side-Channel Attacks on SGX}
%
% TODO: we need to acknowledge cache-timing attacks, as well
% as speculative execution attacks, and mention how these are
% implementation issues that will hopefully be addressed in future
% secure hardware designs.
%Recently, we have seen the rise of side-channel attacks against SGX, including
%the speculative execution attack Foreshadow~\cite{foreshadow,
%weisse2018foreshadow}.  This attack allows for not only the extraction of the
%entire SGX enclave's memory contents but also the attestation and sealing keys.
%We note that this attack would break the security guarantees that we provide
%with conclaves.
%% Given that we use SGX, 
We must address the recent rise
 of side-channel attacks against SGX, including
the speculative execution attack Foreshadow~\cite{foreshadow,
weisse2018foreshadow}.  This attack allows for the extraction of not only the
entire SGX enclave's memory contents but also the attestation and sealing keys.
We note that this attack would break the security guarantees that we provide
with conclaves.
%
Intel has stated that SGX is explicitly designed to not deal
with side-channel attacks in its current state and leaves handling this up to
enclave developers~\cite{sgx-sidechannel, sgx-developers}.
%
Regardless, Intel
has released both microcode patches and recommendations for system level code
that at the current time address Foreshadow and known related attacks
\cite{sgx-patch, canella2018systematic, weisse2018foreshadow}.  There is also
ongoing research to address both speculative execution
as well as other cache-based side-channel attacks on SGX and in general
\cite{yan2018invisispec, oleksenko2018varys, canella2018systematic, shih2017t}.
%
We consider protections against such side-channel attacks to be outside of the
scope of this work and rely on these defenses.
%
%\clg{I can say this right? Or am I waving a flag at a bull?}

%We also note that SGX is explicitly designed to not deal with side-channel
%attacks in its current state and leaves handling this up to enclave developers
%\cite{sgx-sidechannel, sgx-developers}.

% cite Foreshadow, though don't make a big deal about it
% note how this would/would not impact our system
% Foreshadow enables an attacker to read the entire SGX enclave's memory contents.

% Using Foreshadow we have successfully extracted the attestation keys, used by
% the Intel Quoting Enclave to vouch for the authenticity of enclaves. As a
% result, we were able to generate "valid" attestation quotes. Using these
% counterfeit quotes, successfully "proved" to a remote party that a "genuine"
% enclave was running while, in fact, the code was running outside of SGX,
% under our complete control.
% As Foreshadow enables an attacker to extract SGX sealing keys, previously
% sealed data can be modified and re-sealed. With the extracted sealing key, an
% attacker can trivially calculate a valid Message Authentication Code (MAC),
% thus depriving the data owner from the ability to detect the
% modification.
% microcode patches exist --> cite here: https://foreshadowattack.eu/foreshadow-NG.pdf
% there has also been significant research effort on how to mitigate this currently/in the future: cite:
% https://ieeexplore.ieee.org/abstract/document/8574559 --> InvisiSpec
% https://arxiv.org/pdf/1811.05441.pdf --> has a giant list of proposed defenses
% Intel CVE/defense page:
% https://software.intel.com/security-software-guidance/software-guidance/l1-terminal-fault

% http://web.cse.ohio-state.edu/~zhang.834/papers/ccs17b.pdf
