\section{Related Work}
\label{sec:related}

We present an overview of trusted execution environments, and then describe
three, not necessarily orthogonal techniques, for post-hoc adapting existing
software to satisfy alternative trust assumptions. 


\subsection{Trusted Execution Environments}

Trusted execution environments (TEEs) provide hardware protections for running
trusted portions of code with guarantees of confidentiality and integrity.  
%
Applications can be guaranteed that code executed within the TEE was run
correctly and that any secrets generated during execution will remain safely
within it as well.
%
A wide range of TEEs are available today, with varying functionalities.
%
We focus on Intel's Software Guard Extensions (SGX) environment, and AMD's
Secure Encrypted Virtualization, the two TEE implementations for the x86\_64
architecture.


\parhead{Intel SGX Overview}
%
Intel's SGX provides a new mechanism for trusted
hardware and software as an extension to the x86 instruction set~\cite{sgx,
mckeen2013innovative}.  
%
A program called an \textit{enclave} runs at high
privilege in isolation on the processor in order to provide trusted code
execution, while an untrusted application can make calls into the enclave.
%
While these enclaves can be statically disassembled (so the code running in the
enclave is not private), once an enclave is running, its internal state is
opaque to any observer (even one with physical access), as are any secrets generated.  


Enclaves must be measured and signed by their creator and cannot run without
this signature, and the enclave state is checked against this measurement
before running.  
%
An enclave can also cryptographically \textit{attest} to its current state, in
order to prove that it correctly executed code \cite{sgx_provisioning,
anati2013innovative}.  
%
Another feature is the ability to cryptographically \textit{seal} data to be
used across multiple invocations of an enclave~\cite{anati2013innovative,
sgx_sealing}.  
%
SGX also provides such features as trusted time and monotonic counters
\cite{sgx-linux-sdk,sgx-trusted-time}.  
%
However, as enclave code is always executed in user mode, system calls
must be proxied through the untrusted OS.


\parhead{AMD-SEV Overview}
%
Whereas Intel SGX guards fine-grained units of execution, AMD's Secure
Encryped Virtualization (SEV) is roughly the equivalent of an enclaved
virtual machine.
%
SEV is based upon AMD Memory Encryption Technology, which introduces an
AES-128 encryption engine inside the System on Chip (SoC) that transparently
encrypts and decrypts the data when it leaves or enters the SoC, respectively.
%
Encryption key management, such as generating, storing, and delivering the
keys, are carried out by the AMD secure processor and the encryption keys are kept
hidden from untrusted parts of the platform.
%
The secure processor provides a set of APIs for provisioning and managing the
platform in the cloud.


SEV allows individual VMs to encrypt their memroy pages using their own secure
keys; the VM's memory space is thus protected from from the hypervisor and
co-resident VMs.
%
When code and data arrive into the SoC, SEV tags all of the code and data
associated with the guest VM in the cache and limits access only to the tag's
owner VM.
%
The guest OS is able to specify which memory pages are encrypted through new
page table attributes.


\parhead{Attacks}
%
We must address the recent rise of side-channel attacks against SGX, including
the speculative execution attack Foreshadow~\cite{foreshadow,
weisse2018foreshadow}.  
%
This attack allows for the extraction of not only the entire SGX enclave's
memory contents but also the attestation and sealing keys.  
%
We note that this attack would break the security guarantees that we provide
with conclaves.
%
Intel has stated that SGX is explicitly designed to not deal with side-channel
attacks in its current state and leaves handling this up to enclave
developers~\cite{sgx-sidechannel, sgx-developers}.
%
Regardless, Intel has released both microcode patches and recommendations for
system level code that at the current time address Foreshadow and known related
attacks \cite{sgx-patch, canella2018systematic, weisse2018foreshadow}.  
%
There is also ongoing research to address both speculative execution as well as
other cache-based side-channel attacks on SGX and in general
\cite{yan2018invisispec, oleksenko2018varys, canella2018systematic, shih2017t}.

% TODO: add paragraph or two about attacks on SEV.


% Each solution operates at a differnet level in the stack, more or less.
% Handle TEE and non-TEE based solutions together.

\subsection{Partitioning applications across trust domains} 
% here, you can talk about program partitioning, both those that are aware of
% TEEs, and those that came before.


\parhead{Compiler Techniques}
%
Privtrans~\cite{privtrans} partitions a single program into two parts: a
privileged program called the monitor and an unprivileged program called the
slave. 
%
The monitor and slave run as separate processes, but communicate and cooperate
to perform the same function as the original program.
%
This is process-level isolation, and the monitor can be seen as interposing
between privileged operations and the main execution in the slave.


Privtrans allows a developer to add privilege separation to source code via
source-level annotations of variable and functions to indicate privileged
operations.  
%
Privtrans then propagates attributes, performs static analysis to find
privileged call sites, and performs C-to-C translation to partition the input
source code into the source code for the monitor and slave. 
%
Monitor must mediate access to all privileged resources, including the data
derived from such a resource; it is not sufficient for the monitor to only
perform access control.
%
In contrast, developers can augment monitors with application-specific
downgrade policies to make otherwise priviledged data flow from the monitor to
the slaves.  


\parhead{Protocol Refactoring}
%
Other work allows the organization to retain ownership of its private keys by
changing the server-side implementation of TLS\@.
%
SSL Splitting~\cite{ssl-splitting} leverages the fact that a TLS stream
comprises data records and authentication records (MACs), and develops a new
protocol in which the organization sends the authentication records and the
provider merges them with the data records to form the complete TLS stream.
%
Cloudflare's Keyless SSL~\cite{keyless-ssl} takes advantage of the fact that
TLS only uses the website's private key in a single step of the TLS handshake.
%
Like SSL Splitting, Keyless SSL keeps the master private key off of, and unknown
to, the proxy, but unlike SSL Splitting, Keyless SSL does not provide for
content provider endorsement of the content the proxy serves.  
%
Neither SSL splitting nor Keyless SSL provides for the protection of the
session keys from the provider.


\parhead{Virtualization Techniques} 
%
Several works~\cite{overshadow,inktag} that predate Intel SGX and AMD-SEV
extend the isolation capabilties of a hypervisor to give storng safety
guarantees to applications even in the presence of malicious guest operating
system.
%
These systems prevent the guest operating system from reading or modifying
application code, data, and registers through a technique called
\emph{cloaking}, in which the application sees a normal view of its memory
pages, but the OS an encrypted view.
%
Cloaking allows resources to remain accessible to the OS, yet secure,
permitting the OS to manage resources without compromising application
privacy or integrity.
%
Inktag also enables the hypervisor to verify the behavior of the untrusted
commodity operating system, as by checking that the memory mappings requested
by the application match those returned by the OS.


\subsection{Running (trusted) legacy applications in TEEs}
%
Various works use SGX as a mechanism for achieving shielded execution of
unmodified, trusted, legacy applications.
%
These works generally differ in how much of the application's code runs
within the enclave.


\parhead{Minimal TCB approachs}
%
At one extreme, TaLoS~\cite{talos} simply ports the LibreSSL library to SGX so
that the application terminates TLS connections in an enclave; the rest of
the application remains outside the enclave, unchanged.
%
% emphasize: smaller TCB, 
Several works~\cite{glamdring, panoply} generalize this approach by using developer
provided source-level annotations to partition an application into sensitive and
non-sensitive commponts, with the sensitive components running in an enclave.
%
In particular, Glamdring~\cite{glamdring} takes a similar approach to
Privtans~\cite{privtrans}, while also aiding a backslicing technique so that
fucntions and variables that influence a sensitive output are also isolated to
an enclave.  
%
Panoply~\cite{panoply} focuses less on the  partitioning scheme, but instead
allows the application to be partitioned into multiple enclaves, and
enforces run-time data and control flows between the resultant enclaves.


\parhead{Library Operating System Approaches}
%
At the other extreme, SCONE~\cite{scone} moves the entire C library into the enclave.
%
Haven~\cite{haven} and Graphene~\cite{graphene} carry this approach further by
implementing kernel functionality in an enclave by means of a library operating
system (libOS).  libOSes refactor a traditional OS kernel into a user-land
library that loads a program.
%
The program's C library is modified to redirect system calls to the libOS, which
in turn either services the calls internally or calls into the untrusted OS
when the host's resources are needed.
%
Aurora~\cite{liang2018aurora} extends the libOS from the SGX enclave to System
Management Mode (SMM) by running device drivers in SMM memory.


\subsection{Sandboxing (untrusted) applications via information flow controls}

\parhead{OS-Level IFC}
Decentralized information flow control (DIFC) enhanced operating systems allow
dynamic labeling of OS abstrations, such as processes and files, and enforce
access control based on Denning's lattic model for information flow
security~\cite{}.
%
The tags represent different rights (capabilities) with respect to the data
that the process is allowed to read or write.
%
Using tags, these systems can enforce the classical secrecy and integrity policies of
``no read up, no write down", or an integrity policy of ``no read down, no
write up", between the objects in the system.
%
Additional policies, such as \emph{export protection} are possible, in which a
process cannot have both an uncontrolled channel (e.g., a socket) open and
access to private data that it cannot declassify, as well as
policies for stringent system-wide \emph{read-protection} and \emph{integrity
protection}.
%
These process-level information flow models are coarse grained and cannot track
sensitive information \emph{within} untrusted applications.


\parhead{Fine-Grained IFC}
%
Fine-grained information flow tracking (IFT) is a technique whereby execution
runtime tags memory and registers containing sensitive data with labels (also
called taint or taint markings) and propagates these labels in
acccordance with the computation.
%
Instrumentation is transparent to the application process, as the application
observes the same addresses and same values as it would in an uninstrumented
execution.
%
IFT traditionally uses a source and sink model whereby labels are
assigned at sources and checked at sinks; the labels themselves are opaque,
and are interpreted by application-specific policy.
%
Most work in this area has been specifically motivated by detection of control
flow hijacking attacks, such as buffer overflows, and involves tainting of
program inputs, and trapping of control transfers to tainted target addresses.
%
However, IFT is also used to enforce mandatory access control on data and
derived data, and thus a restriction on the flow of information between
the system's objects (e.g., users, processes, files).


Fine-grained IFT can either be applied at the process- or system-level.
%
A standard approach for process-level information flow tracking is to run
the process in an emulator, such as Valgrind, Intel Pin, or DynamoRIO; the
emulator serves as a reference monitor, and dynamically tracks data flows.
%
In contrast, full-system IFT involves running the guest system in a hardware
emulator (e.g., QEMU) that has been augmented with machine instruction analysis
and taint tracking capabilities.
%
As an optimization, and for security isolation, a hypervisor may dynamically
switch execution between emulated execution when processing tainted data, and
native virtualized execution when processing untainted data.
%
Full-system IFT systems also propagate taint to the filesystem (as by storing
taint in a file's extended attributes) and across the network (as by placing
labels in a packet's IP options).


\parhead{Uses with Secure Hardware}
% TODO: also talk about compiler-based techniques (e.g., Moat)
%
Ryoan presents a request-response execution model that allows mutually
distrustful parties to process sensitive data in a distributed fashion on
untrusted infrastructure, without any party leaking secret data.
%
Ryoan runs each party's processing modules in a trusted userspace sandbox,
which in turn is hosted in an SGX enclave.
%
Each module tags its output, essentially performing taint tracking at
enclave-level granularity, where the taint indicates which encalves have seen
secret data.
%
When operating on tainted data, Ryoan ensures that the module cannot leak data
by controlling explicit I/O channels, obfuscating network traffic, forbidding
use of the fileystem, and ensuring that input is only processed once.

