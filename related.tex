\section{Related Work}
\label{sec:related}

\subsection{Intel SGX}
\label{sec:related-sgx}

Intel's SGX provides a new mechanism for trusted hardware and software as an
extension to the x86 instruction set~\cite{sgx, mckeen2013innovative}.  
%
A program called an \textit{enclave} runs at high privilege in isolation on the
processor in order to provide trusted code execution, while an untrusted
application can make calls into the enclave.
%
While these enclaves can be statically disassembled (so the code running in the
enclave is not private), once an enclave is running, its internal state
is opaque to any observer (even one with physical access), as are any secrets
generated.  


Enclaves must be measured and signed by their creator and cannot run without
this signature, and the enclave state is checked against this measurement
before running.  
%
An enclave can also cryptographically \textit{attest} to its current state, in
order to prove that it correctly executed code \cite{sgx_provisioning,
anati2013innovative}.  
%
Another feature is the ability to cryptographically \textit{seal} data to be
used across multiple invocations of an enclave~\cite{anati2013innovative,
sgx_sealing}.  
%
SGX also provides such features as trusted time and monotonic counters
\cite{sgx-linux-sdk,sgx-trusted-time}.  
%
However, an enclaves cannot directly make systems calls, and such calls must be
proxied through the untrusted OS\@.


\parhead{Running Legacy Applications in SGX}
%
Various works use SGX as a mechanism for achieving shielded execution of
unmodified legacy applications.
%
SCONE~\cite{scone} moves the entire C library into the enclave, while
Haven~\cite{haven} and Graphene~\cite{graphene} 
implement kernel functionality in an enclave by means of a library operating
system (libOS).
%
libOSes refactor a traditional OS kernel into a user-land library that loads a
program.
%
The program's C library is modified to redirect system calls to the libOS, which
in turn either services the calls internally or calls into the untrusted OS
when the host's resources are needed.
%
Aurora~\cite{liang2018aurora} extends the libOS from the SGX enclave to System
Management Mode (SMM) by running device drivers in SMM memory.


\parhead{Side Channels}


\subsection{Program Partitioning}

%% TODO: incorporate some of these ideas
% problem definition
%Multiple principals need to cooperate but do not entirely trust one another.
%%
%The general problem with these collaborative computations is ensuring that the
%security policies of all the participants are enforced.
%%
%The goal of secure program partitioning is to ensure that if a host is
%subverted, the only data whose confidentiality or integrity is threatened is
%data owned by principals that have declared they trust; in other words, the
%trusted computing based is localized to trusted hosts.
%%
%Secure program partitioning permits a computation to be described as a single
%program independent of its distributed implementation. 
%%
%The partitioning process then automatically generates a secure protocol for
%data exchange among the hosts.
%%
%Protect the confidentiality of data for computations that manipulate data with
%differing confidentiality needs on an execution platform comprising
%heterogeneously trusted hosts.


% definition and overview
\emph{Program partitioning} is a technique to automatically achieve privilege
separation of a monolithic application by transparently separating the
application into communicating subprograms that collectively implement the
original application.
%
The technique achieves safety by running each subprogram in a separate
isolation domain (a separate process, host, or enclave), and exposing a narrow
interface for communication between the subprograms.
%
Traditionally, a developer indicates the application splits by 
compiler-level annotations or first-class language constructs that specify the
principals involved in a computation, and any restrictions on data flow between
the principals.
%
The compiler then performs static analysis to determine which procedures and
data from the original application are compiled into each subprogram, as well as
generates the RPC glue code for communication between the subprograms.


% paragaph on IFC/taint-tracking, including declassifcation
Central to program partitioning is the concept of \emph{information flow
control} (IFC)---a form of mandatory access control describing which
principals may read or write a given piece of data.
%
In cases where the IFC policy must also apply to derived data, IFC uses a
technique called \emph{taint-tracking}.
%
Taint-tracking is a run-time instrumentation that tracks the
propagation of memory labels (which encode IFC access controls) over the
evolution of a process. 
%
When paired with program partitioning, taint-tracking ensures that sensitive
data does not inadvertently leave a trust boundary.
%
In cases like a signature computation, taint-tracking will incorrectly label
the non-sensitive result (and thus impede progress); for such reasons,
taint-tracking systems allow for data declassification.


% lead-in to single vs.  multi-party
Although program partitioning's original purpose was to reduce the attack
surface of an application, it has more recently found use as a technique for
running applications that involve the joint resources of distrustful parties.
%
I divide the prior work into the specific case of a single-party shielding
some part of the application from an untrusted cloud, and the general case of
multiple parties each shielding some part of the application from one another.
%
I also discuss tangential work that strictly partitions at the protocol layer.


\parhead{Single-party shielding}
%
Arguably the first work in this area is Privtrans~\cite{privtrans}, which
can partition an application such that any authentication mechanism or key
operations occur on a trusted machine, and the remaining portions on a
third-party's machine.
%
In Privtrans, programmers supply source-level annotations of sensitive data and
code.
%
Privtrans then uses static analysis to infer and propagate dependencies between
privileged operations, and a C-to-C translation to partition the source code
into trusted and untrusted programs.
%
Glamdring~\cite{glamdring} extends this idea to enable the trusted
subprogram to execute within an SGX enclave.
%
Glamdring additionally supports data integrity by using \emph{backward
slicing} to identify functions that may affect sensitive data.
%
By placing only the sensitive components of the application in an enclave,
Glamdring presents a minimal TCB alternative to the libOS approaches that
run the entire application within an enclave.
%
GOTEE~\cite{secured-routines} applies these techniques as  a language-specific
abstraction by extending the Go language to execute a \emph{goroutine} within
an enclave, and uses strongly typed channels to communicate between the trusted
and untrusted environments.


%%% GLAMDRING (Extended overview)
% A developer first annotates security-sensitive application data.
%Glamdring~\cite{glamdring} extends this idea to enable the trusted subprogram
%to execute within an enclave.
%%
%To partition an application, a developer first annotates input and output
%variables in the source code that contain sensitive data and whose
%confidentiality and integrity should be protected.
%%
%To preserve data confidentiality, Glamdring uses dataflow analysis to
%identify functions that may be exposed to sensitive data;  for data
%integrity, it uses back ward slicing to identify functions that may affect
%sensitive data. 
%%
%Glamdring then places security-sensitive functions inside the enclave, and adds
%runtime checks and cryptographic operations at the enclave boundary to protect
%it from attack
%%
%By placing only the sensitive components of the application in an enclave,
%Glamdring presents an minimal TCB alternative to the approaches that run the
%entire application within an encalve by use of an SGX-specific library
%operating system.


%%% GOTEE (Extended overview)
%GOTEE~\cite{secured-routines} moves this idea into language-specific
%abstraction by extending the Go language to allow a programmer to execute a
%goroutine within an enclave, and use low-overhead, strongly typed channels to
%communicate bwteen the trusted and untrsuted environments.
%%
%In GOTEE, the compiler to automatically extracts the secure code and data into
%a statically-linked trusted binary.
%%
%Trusted and untrusted domains have their own runtime, memory management, and
%scheduler. 
%%
%GOTEE coordinates interactions between trusted and untrusted code,
%replaces control transfers between these domains with inexpensive
%synchronized data transfers using strongly-typed cross-domain channels.



\parhead{Multi-party}
% -- two paragraphs
%
%
% Jif: original language-based approach
%%%%%%%%%%%%%%%%%%%%%%%%%%%%%%%%%%%%%%%%%%%
Jif/split~\cite{jif} is one of the earliest language-level approaches for using
program partitioning to protect the confidentiality of data within an
application comprising heterogeneously trusted hosts.
%
Jif/split extends Java with explicit program annotations and security types
that specify IFC restrictions, and the compiler and run-time system reject
programs that violate the restrictions. 
%
Civet~\cite{civet} instead focuses on partitioning Java applications into sets of classes
that run inside enclaves and untrusted classes that run outside enclaves; each
set of enclaved classes represents a trust domain.
% TODO: cite sam type-confustion paper as civert
Civet's contributions include type-checking inputs to enclaved
objects (thereby guarding against polymorphic type-confusion), and dynamic
taint-tracking of objects instantiated inside of trusted code or provisioned
from a secure channel, to prevent data leakage.


Ryoan~\cite{ryoan} also uses SGX enclaves, but assumes a directed acyclic graph
execution model where each party receives input from a peer, process the input,
and sends its output to another party.
%
Ryoan runs each party's processing modules in a trusted userspace sandbox,
which in turn is hosted in an enclave.
%
Each module tags its output, essentially performing taint tracking at
enclave-level granularity, where the taint indicates which enclaves have seen
secret data.
%
When operating on tainted data, Ryoan ensures that the module cannot leak data
by controlling explicit I/O channels, obfuscating network traffic, forbidding
use of the fileystem, and ensuring that input is only processed once.


Wysteria~\cite{wysteria} is a high-level language for writing \emph{secure
multi-party computations} (SMCs)---computation schemes that allow
multiple parties to cooperatively compute a function over their private
inputs such that the output is public but the inputs remain private. 
%
SMC protocols are typically implemented using garbled circuits, homomorphic
encryption, or secret shares, and represent an alternative to secure enclaves.
%
Wysteria is noteworthy for supporting \emph{mixed-mode} programs, where the
program can operate in a combination of parallel and secure modes, where the
former identifies local computations taking place on one or more hosts (in
parallel), and the latter identifies secure computations occurring
jointly among parties. 
%
%Mixed-mode computations improve performance over monolithic secure
%computations.


% Civet (incorporation with secure hardware)
%       (extended description)
%%%%%%%%%%%
%Civet is a framework for partitioning Java applications into trusted classes
%that run inside enclaves and untrusted classes that run outside enclaves.
%%
%To create a partition, Civet requires developers to identify one or more entry
%classes within the application, to serve as the interface be tween enclave
%code and non-enclave code. A set of entry classes define a trusted domain, in
%which all of the classes that implement the enclave functionality are mutually
%trusted. 
%%
%Every call from an untrusted class to an entry class transitions
%execution into the enclave. After defining the entry classes, the developers
%can specify extra shield classes that leverage object orientation to wrap the
%entry classes. 
%%
%Shield classes are primarily used for tasks such as sanitizing
%or decrypting inputs, or encrypting outputs. 
%%
%Developers can write a shield class without changing the source code of the
%original application
%%
%For each entry class, Civet synthesizes a proxy class that marshals inputs to
%the enclave and invokes an RPC to code running in the enclave.
%%
%Civet also synthesizes extra protections, including type-checking inputs
%(important due to polymorphsim concerns) and
%dynamic taint-tracking of objects instatnited inside of trsuted code or
%provisioned from a secure channel.



% Ryoan (as an example of using secure hardware, but with a different
% programming model) (extended descriptoin)
%%%%%%%%%%
% 
% Ryoan presents a request-response execution model that allows mutually
% distrustful parties to process sensitive data in a distributed fashion on
% untrusted infrastructure, without any party leaking secret data.
% %
% The request-response model means that the global computation takes the form of
% a directed acyclic graph, where a module receives input from a peer module,
% processes the input, and sends its output to another module.
% %
% Ryoan runs each party's processing modules in a trusted userspace sandbox,
% which in turn is hosted in an SGX enclave.
% %
% Each module tags its output, essentially performing taint tracking at
% enclave-level granularity, where the taint indicates which enclaves have seen
% secret data.
% %
% When operating on tainted data, Ryoan ensures that the module cannot leak data
% by controlling explicit I/O channels, obfuscating network traffic, forbidding
% use of the fileystem, and ensuring that input is only processed once.



\parhead{Protocol-level partitioning}
%
In light of the challenge of composing  TLS with HTTP proxies, several lines of
work instead partition the TLS protocol to enable origin servers to retain
ownership of their private keys.
%
SSL Splitting~\cite{ssl-splitting} leverages the fact that a TLS stream
comprises data records and authentication records (MACs), and develops a new
protocol in which the origin sends the authentication records and the proxy
merges them with the data records to form the complete TLS stream. 
%
Cloudflare’s Keyless SSL~\cite{keyless-ssl} takes advantage of the fact that
TLS only uses the website’s private key in a single step of the TLS handshake.
%
Like SSL Splitting, Keyless SSL keeps the master private key off of, and
unknown to, the proxy, but unlike SSL Splitting, Keyless SSL does not provide
for content provider endorsement of the content the proxy serves.



% TODO: note sure where flume, DTA, and Intel Pin will fit in; perhaps
% when suggesting possible codomain implementations
%
% futhermore, if you mention memory keys in the codomain implementation, then
% cite shreds and domenclave.



