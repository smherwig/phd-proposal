\section{Classic}
% todo: I haven't read all of these; make sure they are appropriate.
% could add: iago attacks (aspolos 2013)

\begin{enumerate}
\item  Saltzer, J., Reed, D., and Clark, D. End-to-end arguments in system design. ACM Transactions on Computer Systems 2, 4 (Nov. 1984).
    % \cite{e2e-argument}
\item Liedtke, J. On micro-kernel construction. In ACM Symposium on Operating Systems Principles (SOSP) (1995).
    % \cite{micro-kernel-construction}
\item  Pike, R., Presotto, D., Dorward, S., Flandrena, B., Thompson, K., Trickey, H., and Winterbottom, P. Plan 9 from Bell Labs, 1995.
    % \cite{plan9}
\item Douglis, F., and Ousterhout, J. Transparent process migration: Design
    alternatives and the sprite implementation. Software—Practice \& Experience 21, 8 (1991), 757–785.
    % \cite{transparent-process-migration}
\item  Lampson, B. W. Hints for computer system design. In ACM Symposium on Operating Systems Principles (SOSP) (1983).
    % \cite{hints-for-computer-system-design}
\item Engler, D. R., Kaashoek, M. F., and O’Toole, Jr., J. Exokernel: An operating system architecture for application-level resource management. In ACM Symposium on Operating Systems Principles (SOSP) (1995).
    % \cite{exokernel}
\item  Bershad, B. N., Anderson, T. E., Lazowska, E. D., and Levy, H. M. Lightweight remote procedure call. In ACM Symposium on Operating Systems Principles (SOSP).
	% \cite{lightweight-rpc}
\item  Barham, P., Dragovic, B., Fraser, K., Hand, S., Harris, T., Ho, A., Neugebauer, R., Pratt, I., and Warfield, A. Xen and the art of virtualization. ACM SIGOPS Operating Systems Review 37, 5 (2003), 164–177.
	%\cite{barham2003xen}
\item Thompson, K. Reflections on trusting trust. Communications of the ACM 27, 8 (1984), 761–763.
    % \cite{reflections-on-trusting-trust}
\item Chandy, K. M., and Lamport, L. Distributed snapshots: Determining global states of distributed systems. ACM Transactions on Computer Systems 3, 1 (1985), 63–75.
    % \cite{distributed-snapshots}
\end{enumerate}


\section{Secure/Isolated Execution}
% could add: \cite{scone}
% could remove: \cite{overshadow}, \cite{flicker}, \cite{lwcs}

\begin{enumerate}[resume]
%\setcounter{enumi}{1}
\item Costan, V., and Devadas, S. Intel SGX Explained. Tech. Rep. 2016/086, Cryptology ePrint Archive, 2016.
    % \cite{intel-sgx-explained}
\item  Tsai, C.-C., Porter, D. E., and Vij, M. Graphene-SGX: A practical library OS for unmodified applications on SGX. In USENIX Annual Technical Conference (2017).
	%\cite{graphene}
\item Van Bulck, J., Minkin, M., Weisse, O., Genkin, D., Kasikci, B., Piessens, F., Silberstein, M., Wenisch, T. F., Yarom, Y., and Strackx, R. Foreshadow: Extracting the keys to the Intel SGX kingdom with transient out-of-order execution. In 27th USENIX Security Symposium (USENIX Security 18) (2018), pp. 991–1008.
	%\cite{foreshadow}
\item Hunt, T., Zhu, Z., Xu, Y., Peter, S., and Witchel, E. Ryoan: A distributed sandbox for untrusted computation on secret data. In Proceedings of the 12th USENIX Conference on Operating Systems Design and Implementation (2016), Symposium on Operating Systems Design and Implementation (OSDI).
    %\cite{ryoan}
\item Chen, X., Garfinkel, T., Lewis, E. C., Subrahmanyam, P., Waldspurger, C. A., Boneh, D., Dwoskin, J., and Ports, D. R. Overshadow: A virtualization-based approach to retrofitting protection in commodity operating systems. In ACM International Conference on Architectural Support for Programming Languages and Operating Systems (ASPLOS) (2008).
	% \cite{overshadow}
\item McCune, J. M., Parno, B. J., Perrig, A., Reiter, M. K., and Isozaki, H. Flicker: An execution infrastructure for tcb minimization. In European Conference on Computer Systems (EuroSys) (2008).
    %\cite{flicker}
\item  Chen, Y., Reymondjohnson, S., Sun, Z., and Lu, L. Shreds: Fine-grained execution units with private memory. In IEEE Symposium on Security and Privacy (2016).
	% \cite{shreds}
\item  Litton, J., Vahldiek-Oberwagner, A., Elnikety, E., Garg, D., Bhattacharjee, B., and Druschel, P. Light-weight contexts: An os abstraction for safety and performance. In Symposium on Operating Systems Design and Implementation (OSDI) (2016).
    % \cite{lwcs}
\end{enumerate}

\section{Program Partitioning} 
% could add: \cite{panoply}, \cite{enclavedom}

\begin{enumerate}[resume]
\item  Brumley, D., and Song, D. Privtrans: Automatically partitioning programs for privilege separation. In USENIX Security Symposium (2004).
	% \cite{privtrans}
\item  Lind, J., Priebe, C., Muthukumaran, D., O’Keeffe, D., Aublin, P.-L., Kelbert, F., Reiher, T., Goltzsche, D., Eyers, D., Kapitza, R., Fetzer, C., and Pietzuch, P. Glamdring: Automatic application partitioning for Intel SGX. In USENIX Annual Technical Conference (2017).
	% \cite{glamdring}
\item  Zdancewic, S., Zheng, L., Nystrom, N., and Myers, A. C. Untrusted hosts and confidentiality: Secure program partitioning. In ACM Symposium on Operating Systems Principles (SOSP) (2001).
    % \cite{jif}
\item  Rastogi, A., Hammer, M. A., and Hicks, M. Wysteria: A programming language for generic, mixed-mode multiparty computations. In IEEE Symposium on Security and Privacy (2014).
    %\cite{wysteria}
\item Ghosn, A., Larus, J. R., and Bugnion, E. Secured routines: Language-based construction of trusted execution environments. In USENIX Annual Technical Conference (2019).
    % \cite{secured-routines}
\item Bittau, A., Marchenko, P., Handley, M., and Karp, B. Wedge: Splitting applications into reduced-privilege compartments. In Symposium on Networked Systems Design and Implementation (NSDI) (2008).
    % \cite{wedge}
\item Tsai, C.-C., Son, J., Jain, B., McAvey, J., Popa, R. A., and Porter, D. E. Civet: An efficient Java partitioning framework for hardware enclaves. In USENIX Security Symposium (2020).
    % \cite{civet}
\item Rubinov, K., Rosculete, L., Mitra, T., and Roychoudhury, A. Automated partitioning of Android applications for trusted execution environments. In International Conferenceon Software Engineering (2016).
    % \cite{partitioning-android-apps}
\item Santos, N., Raj, H., Saroiu, S., and Wolman, A. Using arm trustzone to build a trusted language runtime for mobile applications. In ACM International Conference on Architectural Support for Programming Languages and Operating Systems (ASPLOS) (2014).
    % \cite{arm-trusted-language-runtime}
\end{enumerate}

\section{Information Flow Control and Taint Tracking}
% could remove: \cite{pin-pldi}, \cite{libdft}, \cite{taint-demand-emulation}

\begin{enumerate}[resume]
\item Luk, C.-K., Cohn, R., Muth, R., Patil, H., Klauser, A., Lowney, G., Wallace, S., Reddi, V. J., and Hazelwood, K. Pin: Building customized program analysis tools with dynamic instrumentation. In ACM SIGPLAN Conference on Programming Language Design and Implementation (2005).
    %\cite{pin-pldi}
\item  Newsome, J., and Song, D. Dynamic taint analysis for automatic detection, analysis, and signature generation of exploits on commodity software. In Network and Distributed System Security Symposium (NDSS) (2005).
	%\cite{taint-tracking}
\item Kemerlis, V. P., Portokalidis, G., Jee, K., and Keromytis, A. D. libdft: Practical dynamic data flow tracking for commodity systems. In ACM International Conference on Virtual Execution Environments (VEE) (2012).
	%\cite{libdft}
\item Ho, A., Fetterman, M., Clark, C., Warfield, A., and Hand, S. Practical taint-based protection using demand emulation. In European Conference on Computer Systems (EuroSys) (2006).
    % \cite {taint-demand-emulation}
\item Krohn, M., Yip, A., Brodsky, M., Cliffer, N., Kaashoek, M. F., Kohler, E., and Morris, R. Information flow control for standard OS abstractions. In ACM Symposium on Operating Systems Principles (SOSP) (2007).
    % \cite{flume}
\item Enck, W., Gilbert, P., Chun, B.-G., Cox, L. P., Jung, J., McDaniel, P., and Sheth, A. N. Taintdroid: An information-flow tracking system for realtime privacy monitoring on smartphones. In Symposium on Operating Systems Design and Implementation (OSDI) (2010).
    % \cite{taintdroid}
\end{enumerate}
