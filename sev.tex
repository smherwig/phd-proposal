\section{AMD Memory Encryption Technology}

% References: https://en.wikichip.org/wiki/x86/sme
% https://lwn.net/Articles/716165/
%
%


% Threat Model
%%--------------------
AMD Memory Encryption Technology addresses two difference classes of attackds:
system software level and physical access attcks.  
%
System software levle attacks  icnludes a higher-privileged entity, such as a
hypervisor, taht analyzes the guest VM memory space for malacious purposes ofr
for edeployg attakcs that use hyperviosr vulneratlibity to apply side-channel
attacks to toher co-resident guest VMS (cross-VM attackes).
%
Physical attacks include hot memory I/Ot attacsk or col boot attacks.


% Design
%-------------------------
AMD Memory Encryption Technology introdues an AES-128 encryption engine insde
the System on Chip (SoC) that tansparently encrypted and decrypts the wahat
when it leeavr or tenters the SoC, respectively.
%
AMD proposes two main features based on Memory Encryption Technology: (1)
Secure Memory Encryption (SME) and (2) Secure Encryption Virtualizaiotn (SEV).
%
Both SEV and SME are manged by the OR or hypervisor, and no applciaton software
chagnes are needed.
%
AMD's Memory Encryption Technology does not provide memory integrity
protection.


% AMD secure coprocessor
%-------------------------
Encryption key mangmenet, such as generating, storign, and delivering the keys
are carried out by the AMD secure processor and the encryption keys are kept
hidden from untursted parts of the platform.
%
The AMD secure processor utilaizes a 32-bit ARM Cortex A5, and uses its memory
an storage while executing a kernel that is signed by AMD


% Secure Memory Encryption (SME), and 
% Transparent Secure memory Encryption (TSME)
%---------------------------------------
SEM is teh security feature that addresses physical access atacks.
%
SEM is an x86 insruction set extension introduced by AMD for page-granulaor
memory encryton support using a single eephmeral key.

It uses an encryptoin key that is randomly generated by teh AMD seucr processos
and is loaded into the meory controller at boot toime to encrypt the memory.
%
The OS is able to leverage the SME by setting a bit in the x86 page table (teh
en\emph{C}rypted, or \emph{C-bit})
%
When the C-bit is set, accss to that memory page is directed to the AMD Memory
Encryption Engine.
%
In the SME design, all devices can access the encrypted pages through DMA.


In the cases of legacy OSes, a feature called Transparent Secure memory
Encryption (TSME) allows the C-bit to effectively be set in the BIOS.


% Secure Encrypted Virtualization 
%---------------------------------------
SEV extens SME to AMD-V, allowing individual VMs to run SEM using their own
secure keys.
%
SEV is a security feature that mainly addresses the high-privileged system
software class of attacked by providing encrypted VM isolation.
%
SEV encrypts and protects the VM's memory space from the hypervisor and
co-resident VMS using a VM-specific encryption key.
%
When code and data arrives into the SoC, SEV tags all of th e code and data
associated with the guest VM in the cache and limites access only to the tag's
owner VM.

Under SEV, the ASI field in teh page table is used as teh key index that
identifies which encryption key is ued to encrypt and decrypt the memroy
traffic asscoated with that VM.
%
SEV-enabled VMs can control their own C-bit for memory pages they want
encrypted.


% Attestation and Provisioning
%------------------------------
The AMD secur processor provides a set of APIs for provisioning and managing
the platform in the cloud.
%
The hypervisor's SEV driver can invoke these APIs.
%
In the SEV architectures, a guest owner manages her guest ecret and generates
teh policies for VM migration of debuggin.
%
The Diffie-Hellman key exhcnage protocl is ued between the guest owner and the
AMD seucre processor to open a channgel between the guest owner and the AMD
secure processor; the guest owner is enabled to autehnticat teh seucre
processor and exchagne informaiton to set up the protected VM.
%
The guest owner can set a policy where certain pages are private and others
shared; if shared, the hyperviosr can read it in plain text.

SEV may be sued in conjection with SME.  Under this scenario, each SEV-enabled
VM controls its own encryption via the C-bit and the hsot page tables control
the encryption for shared memory.
























