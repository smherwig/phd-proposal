\section{AMD Memory Encryption Technology}

% References: https://en.wikichip.org/wiki/x86/sme
% https://lwn.net/Articles/716165/
%
%


% Threat Model
%%--------------------
AMD Memory Encryption Technology addresses two difference classes of attacks:
system software level and physical access attacks.  
%
System software level attacks include a higher-privileged entity, such as a
hypervisor, that analyzes the guest VM memory space for malacious purposes or
for deploygin attacks that use a hypervisor vulnerability to apply side-channel
attacks to other co-resident guest VMs (cross-VM attackes).
%
Physical attacks include hot memory I/O attacks or cold boot attacks.


% Design
%-------------------------
AMD Memory Encryption Technology introdues an AES-128 encryption engine inside
the System on Chip (SoC) that transparently encrypts and decrypts the data
when it leeaves or enters the SoC, respectively.
%
AMD proposes two main features based on Memory Encryption Technology: (1)
Secure Memory Encryption (SME) and (2) Secure Encryption Virtualizaiotn (SEV).
%
Both SEV and SME are managed by the OS or hypervisor, and no application software
changes are needed.
%
AMD's Memory Encryption Technology does not provide memory integrity
protection.


% AMD secure coprocessor
%-------------------------
Encryption key management, such as generating, storing, and delivering the
keys, are carried out by the AMD secure processor and the encryption keys are kept
hidden from untrusted parts of the platform.
%
The AMD secure processor utilaizes a 32-bit ARM Cortex A5, and uses its memory
an storage while executing a kernel that is signed by AMD


% Secure Memory Encryption (SME), and 
% Transparent Secure memory Encryption (TSME)
%---------------------------------------
SEM is the security feature that addresses physical access atacks.
%
SEM is an x86 insruction set extension introduced by AMD for page-granulaor
memory encryption support using a single ephemeral key.
%
The AMD secure processor randomly generates the key
and loads it into the meory controller at boot time.
%
The OS is able to leverage the SME by setting a bit in the x86 page table (the
en\emph{C}rypted, or \emph{C-bit})
%
When the C-bit is set, accss to that memory page is directed to the AMD Memory
Encryption Engine.
%
In the SME design, all devices can access the encrypted pages through DMA.


In the cases of legacy OSes, a feature called Transparent Secure memory
Encryption (TSME) allows the C-bit to effectively be set in the BIOS for all
memory pages.


% Secure Encrypted Virtualization 
%---------------------------------------
SEV extends SME to AMD-V, allowing individual VMs to run SEM using their own
secure keys.
%
SEV is a security feature that mainly addresses the high-privileged system
software class of attacks by providing encrypted VM isolation.
%
SEV encrypts and protects the VM's memory space from the hypervisor and
co-resident VMs using a VM-specific encryption key.
%
When code and data arrives into the SoC, SEV tags all of the code and data
associated with the guest VM in the cache and limites access only to the tag's
owner VM.

Under SEV, the ASID field in the page table is used as the key index that
identifies which encryption key is ued to encrypt and decrypt the memory
traffic associated with that VM.
%
SEV-enabled VMs can control their own C-bit for memory pages they want
encrypted.


% Attestation and Provisioning
%------------------------------
The AMD secure processor provides a set of APIs for provisioning and managing
the platform in the cloud.
%
The hypervisor's SEV driver can invoke these APIs.
%
In the SEV architectures, a guest owner manages her guest secret and generates
the policies for VM migration of debugging.
%
The Diffie-Hellman key exchange protocol is used between the guest owner and the
AMD secure processor to open a channel between the guest owner and the AMD
secure processor; the guest owner is able to authenticate the secure
processor and exchange information  to set up the protected VM.
%
The guest owner can set a policy where certain pages are private and others
shared; if shared, the hypervisor can read it in plain text.

SEV may be used in conjunction with SME.  Under this scenario, each SEV-enabled
VM controls its own encryption via the C-bit and the host page tables control
the encryption for shared memory.
























