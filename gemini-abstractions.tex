\section{Gemini Core Components}
\label{sec:gemini-abstractions}


\subsection{Co-processes}

A co-process is a userspace monitor that controls the resources and
privileges of a client process
%
Abstractly, the monitor and client are connected with pipe and may perform
generic IPC\@.
%
However, unlike a pipe, a client may yield to the monitor; the client's thread
is suspended, and the monitor may modify the thread's execution state.
%
The monitor may resume the client process, perform, paging,
interpose on system calls, apply network filters, or force the client to yield.
%
A client process may have only a single monitor, but a monitor may be the
co-process for multiple clients.
%
The features of a co-process are a mix of ptrace, userfaultfd, and a userspace
Linux Security Module (LSM).  


%\begin{table*}
%\resizebox{\textwidth}{!}{
%
%\centering
%\rowcolors{2}{gray!15}{white}
%    \begin{tabular}{@{}ll@{}}
%        \textbf{Function}        & \textbf{Description} \\
%        \hline
%        \texttt{coproc\_create} & foo bar baz foo bar baz foo bar baz foo bar baz \\
%        \texttt{coproc\_resume} & \\
%        \texttt{coproc\_control}& \\
%        \texttt{coproc\_attach} & \\
%        \texttt{coproc\_yield}  & \\
%        \texttt{coproc\_call}   & \\
%\end{tabular}
%
%}
%\caption{libcoproc API}
%\label{tab:libcoproc-api}
%\vspace*{-10pt}
%\end{table*}

\parhead{Creating a co-process}

\begin{lstlisting}[frame=single,caption={Monitor runloop},captionpos=b]
monfd = coproc_create(monitor_id)
listen(monfd)
fdset.add(monfd)

while True:
    poll(fdset)
    if monfd.ready:
        fd = accept(monfd)
        fdset.ad(fd)
    else:
        msg = readmsg(fd)
        if msg.event == YIELD:
            # do something
            coproc_resume(fd)
        else:
            # do something
\end{lstlisting}


\parhead{Attaching to a co-process}

\begin{lstlisting}[frame=single,caption={Launching a program under a monitor},captionpos=b]
fd = coproc_attach(monitor_id)
execve(program)
\end{lstlisting}


\parhead{Page Fault Handling}


\parhead{Descriptor Proxying}


\parhead{Security Decisions}


\subsection{Distributed Container}

TODO: describe and give pseudo-code for building a distributed container
abstraction out of the migration and pinning abstractions.

%Gemini migrates at the thread-level.
%%
%When a thread attempts to access a resource pinned to a target domain,
%the system monitor suspends the thread, and transfers the
%thread's execution to the target.
%%
%The thread runs on the target until it satisfies a \emph{migration condition},
%and then migrates back to the source.
%%
%On return, the source updates the thread's context and the process's memory
%space, and resumes the thread.


\parhead{Distributed Shared Memory}

TODO: describe and give pseudo-code for building a DSM abstrction on top of the
coproc abstraction.

%Similar to prior systems, Gemini integrates distributed shared memory (DSM) and
%virtual memory management.
%%
%Each domain runs a DSM manager and page server that cooperates with the VM
%manager to service page faults.
%%
%The VM manager refers remote memory access to the DSM manager,
%which in turn satifies the request using a coherence protocol with the peer
%domain's page server.
%%
%Thus, a page fault to local memory is indistinguishable from a page fault to
%remote memory.

\parhead{Memory and File Pinning}

TODO: describe and give pseudo-code for building a memory and file pinning
abstraction on top of the DSM abstraction and file extended attributes.

%\parhead{Distributed Filesystem}
%%
%Gemini maintains a consistent filesystem view among migrated and
%non-migrated processes by allowing a thread's file descriptor table on the
%target to proxy calls to a fileserver back on the source.
%%
%This has additional benefit of obviating the need to replicate or update
%public files during migration.
%%
%Domains may pin file descriptors, such that the descriptor can only be used
%in that domain.
%%
%Likewise, if the target thread writes tainted data to a remote descriptor, the
%associated file transfers to, and becomes pinned by, the target; any processes
%on source with that file open have subsequent operations on that descriptor
%proxied to the target.

\subsection{Memory and File Labeling}

TODO: describe what this abstraction looks like
Describe how policy modules are built on top of memory and file labeling and
the coproc abstraction.


\subsection{Migration}

TODO: describe what this abstraction looks like


