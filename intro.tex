\section{Introduction}
\label{sec:intro}
%    What problems are you trying to solve
%    carve out the space
%
%   We need cover concepts/problems that are common to all
%   the deployment stragegies, as well as define common terms here.


\subsection{Content Delivery Networks} % {{{

CDNs are third-party services that host their customers' websites (and
other data).
%
Virtually all of the most popular websites (and a very long tail of
unpopular websites) use one or more CDNs to help reliably host their
content~\cite{key-sharing}.
%
Historically, CDNs have been thought of as a massive web
cache~\cite{cdn-on-demand}, but today's CDNs play a critical role in
achieving the performance and security that the web relies
on~\cite{securing-cdns}.


We identify four key roles that fundamentally define today's CDNs, and
their enabling technologies:

\parhead{Low latency to clients:} % {{{
%
The primary driving feature of CDNs is their ability to offer low
page-load times for clients visiting their customers' websites.
	
\smallskip\noindent
%
\emph{How they achieve this:}
%
CDNs achieve low latencies via a massive, global network of
\emph{multi-tenant edge servers}.
%
Edge servers act primarily as reverse proxy web servers for the CDN's
customers: to handle client requests, edge servers retrieve content
from the customers' \emph{origin servers}, and cache it so they can
deliver it locally.
%
CDNs direct client requests to the edge servers in a way that balances
load across the servers, and that minimizes client latency---often by
locating the ``closest'' server to the client.
%
There are many sophisticated means of routing clients to nearby
servers, involving IP geolocation, IP anycast, and DNS load
balancing.


Edge-network services like CDNs therefore derive much of their utility
from the fact that they have servers close to most clients.
%
To this end, CDNs deploy their own data centers, and deploy servers
within other organizations' networks, such as college campuses, ISPs,
or companies.
%
Indeed, today's CDNs have so many points of presence (PoPs) that they
often are within the \emph{same} network as the clients visiting their
sites.
%
To support such proximity without an inordinate number of machines,
CDNs rely on the ability to host multiple tenants (customers) on
their web servers at a time.

% }}}

\parhead{Manage customers' keys:} % {{{
%
As the web moves towards
HTTPS-everywhere~\cite{felt-2017-https}, customers increasingly
rely on CDNs to store their HTTPS certificates and the corresponding
secret keys, so that they can accept TLS connections while maintaining
low latency to clients.


\smallskip\noindent
%
\emph{How they achieve this:}
%
CDNs manage their customers' keys in a variety of ways: sometimes
by having their customers upload their secret keys, but typically by
simply generating keys and obtaining certificates on their customers'
behalf~\cite{key-sharing,when-https-meets-cdn}.
%
Many CDNs combine multiple customers onto single ``cruiseliner
certificates'' under the same key pair---these customers are not allowed
to access their own private keys, as that would allow them to
impersonate any other customer's website on the same cruiseliner
certificate~\cite{key-sharing}.
%
A recent protocol, Keyless SSL~\cite{keyless-ssl}, has been proposed
to address this; we describe this in more detail in \S\ref{sec:prior}.

% }}}

\parhead{Absorb DDoS traffic:} % {{{
%
CDNs protect their customers by filtering DDoS traffic, keeping it from
reaching their customers' networks.

\smallskip\noindent
%
\emph{How they achieve this:}
%
CDNs leverage economies of scale to obtain an incredible amount of
bandwidth and computing resources.  Their customers' networks block
most inbound traffic, except from the CDN\@.
%
Thus, attackers must overcome these huge resources in order to impact a
customer's website.

% }}}

\parhead{Filter targeted attacks:} % {{{
%
An often overlooked but critical feature~\cite{securing-cdns} of
today's CDNs is the ability to filter out (non-DDoS) attack traffic,
such as SQL injection and cross-site scripting attacks.


\smallskip\noindent
%
\emph{How they achieve this:}
%
Unlike with DDoS traffic, the primary challenge behind protecting
against targeted attacks is \emph{detecting} them.
%
CDNs achieve this by running \emph{web-application firewalls (WAFs)},
such as ModSecurity~\cite{modsecurity}.
%
WAFs analyze the plaintext HTTP messages, and compare the messages against
a set of rules (often expressed as regular expressions~\cite{owasp})
that indicate an attack.
%
Edge servers only permit benign data to pass through to the customer's
origin server.



\subsection{Security Implications of CDNs} % {{{

Simultaneously fulfilling these four roles---low latency, key
management, absorbing large attacks, and blocking small
attacks---inherently requires processing client requests on
edge servers.
%
In the presence of HTTPS, however, this processing requires
edge servers to have at least each TLS connection's session key, if not
also each customer's private key.

It is therefore little surprise that CDNs have amassed the vast
majority of private keys on the
web~\cite{key-sharing,when-https-meets-cdn}.
%
This has significant implications on the trust model of the PKI and the
web writ large: today's CDNs could arbitrarily impersonate any of their
customers---and recall that virtually all of the most popular websites
use one or more CDNs~\cite{key-sharing}.

Even if one were to assume a trustworthy CDN, the need to store
sensitive key materials on edge servers introduces significant
challenges.
%
CDNs have historically relied on a combination of their own physical
deployments and deployment within third-party networks, such as college
campuses.
%
To protect their customers' keys, some CDNs refuse to deploy
HTTPS content anywhere but at the data centers they have full physical
control over~\cite{securing-cdns}.
%
However, as the web moves towards HTTPS-everywhere, this means that
such CDNs can no longer make as much use out of third-party networks.
%
In short, without additional protections for private and session keys
on edge servers, the move towards HTTPS-everywhere represents an
\emph{existential threat} to edge-network services.

% }}}

\subsection{Our Goals} % {{{
\label{sec:goals}

At a high level, our goal is to maintain all of the core properties of
a CDN---low latency, key management, and resilience to DDoS and
targeted attacks---without having to expose customers' keys or client's
sensitive information, and without requiring massive code changes from
their customers.
%
We distill our overarching goal down to five specifics:

\vspace*{-0.3\baselineskip}
\begin{enumerate}
\setlength{\itemsep}{0pt}
	\item \textbf{Protect private keys:} Support HTTPS, but without
		exposing the private keys corresponding to the certificate's
		public key to any edge server.
	\item \textbf{Protect session keys:} Once a connection is
		established, do not expose the ephemeral session keys (nor the
		sensitive material for session resumption) to any edge server.
	\item \textbf{Secure web-application firewalls:} Support
		edge-server-side WAFs, but without leaking plaintext messages
		to the server.
	\item \textbf{Support multi-tenancy:} Be able to host
		multiple customers on a single machine (or even the same web
		server process), but with strong isolation between them.
	\item \textbf{Support legacy customer applications:} Support all of
		the same web architectures of today, with minimal modifications
		to or impact on customer code.
\end{enumerate}
\vspace*{-0.3\baselineskip}

These goals are a departure from today's CDNs, which
store all of their customers' keys (at least the session
keys), and operate on the plaintext of the client's data.
%
Achieving these goals stands to improve websites' security, users'
privacy, and also the flexibility in how edge-network services
can be deployed.
