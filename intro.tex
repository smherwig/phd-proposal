\section{Introduction}
\label{sec:intro}

%% big picture
Applications often assume that a single party controls the application, its
data, and its execution environment.
%
This assumption is valid when a party develops the software, owns the data, 
and administers the deployment infrastructure.
%
However, when these applications must run under a different deployment, such as
on a third-party's cloud, or use a heterogenenous dataset, such as data from
multiple parties, the application's original assumption of a monotlithic trust
setting is no longer valid.


%% statement of problem (why big picture murky)
Unfortunately, the operational \emph{status quo} is either to ignore the issue,
or restrict the scope of such applications.  
%
As a result, organizations outsource their applications to the cloud, despite
exposing private data about itself and its patrons to the cloud operator, while
data for healthcare providers and financial services remains confined to data
islands, despite the important questions that they could answer in aggregate.



%% challenges in solving it    (why does it remain murky)
Academia and industry have devoted considerable attention to the underlying
problems of \emph{secure remote computation} and \emph{secure multi-party
computation}.
%
A major difficulty is that conventional operating systems and
programming languages offer limited support for developing and running software
that has an execution model based on heterogeneous trust.
%
Thus, much of the systems-oriented research in this area explores languages,
tools, and system designs for addressing and expressing this execution model.


%%  your insight
My insight is to apply ideas from alternative operating system
architectures that emphasize a distributed and application-specific execution
environment.
%
Whereas the original motivations for these architectures were stability,
performance, or resource utilization, my focus instead is on using them as a
means for constructing and enforcing isolation boundaries that reflect the
application's principals and their trust assumptions.
%
Since the execution enviornment is transparent to the application, 
it may be modified, partitioned, or distributed across domains of varying
trustworthiness, so as to reflect multi-party security goals.
%
Transparency with respect to the application lends this approach to the
important and practical property of enabling \emph{post-hoc} refinements to an
application's security and privacy guarantees, without changing the
application's source code.
%%  your thesis
To that end, my thesis is:
\begin{displayquote}
    It is possible to run legacy application binaries with confidentiality and 
    integrity guarantees that reflect a multi-party trust setting.
\end{displayquote}


%% contributions you make
%Several factors strongly influence the design of such an execution environment:
%the presence of trusted hardware, the trust each party has in the application,
%and the mechanisms for defining and enforcing trust boundaries.
%
In my preliminary work, \textbf{conclaves}, I assume the availability of Intel
SGX secure hardware enclaves.
%
Conclaves extends prior work on SGX-based library operating systems (libOS) by
extending a libOS to support a broader set of legacy services: namely,
multi-process, shared resource, applications.
%
My extensions result in a distributed system reminiscent of a microkernel,
where each kernel service (for instance, a filesystem or shared memory)
runs in a separate enclave and mediates that service’s shared resources
among the application's enclaved processes.
%
With conclaves, the trust policy is code-centric, as the parties specify which
processes comprise the system, as well as which processes may interact with one
another.


My proposed work, \textbf{codomains}, treats trusted hardware availability as a
policy concern.
%
Codomains maintain the source-level abstraction of a monolithic program, but
allow applications to dynamically switch execution to different domains---hosts
and enclaves---via language-neutral mechanisms.
%
Within a domain, the application may pin data to the domain or to a subset of
peer domains.
%
Although codomains present an explicit API for domain switches and data
pinning, the design allows a monitor to implicitly and transparently call such
functions on the application's behalf.
%
In its extreme, codomains allow the trust policy to be data-centric, with parties
specifing which data must be isolated from other parties and how data may flow
throughout the system.


%% description of the rest of paper, at a high-level
This proposal is organized as follows: in \S\ref{sec:background}, I describe
the problem setup, the threat model, and the system goals.
%
In \S\ref{sec:related}, I provide an overview of Intel SGX  and techniques for
running legacy applications in SGX\@.
%
I also describe prior approachs to partitioning an application for the
purposes of separate of privileges, some of which involve separation
into enclave and non-enclave components.
%%
In \S\ref{sec:conclaves-summary}, I summarize my preliminary work.
%
In \S\ref{sec:codomains-intro}, I introduce the concept of codomains, and in
\S\ref{sec:codomains-design} sketch their design.
%
I describe a set of evaluations for codomains in \S\ref{sec:codomains-eval},
and conclude in \S\ref{sec:conclusion}.


