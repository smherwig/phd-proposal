\section{Tor Hidden Services}
\label{sec:tor}
%    What problems are you trying to solve
%    carve out the space

We take a deep look at Tor's onion services (previously called hidden
services).  
%
Our central premise is that the onion service protocols and
best practices are inflexible and restrict the deployment of such services.
%
Specifically, there are few options for an onion service operator to customize
and modulate the availability, performance, privacy guarantees, or usability of
their service.
%
These restrictions stand in stark contrast to the myriad deployment options
available to a service operator on the normal Internet.


We first describe how onion services work.
%
We then validate our premise: we identify the key pain points in deploying an
onion service, and describe various approaches that the Tor community has
proposed and implemented to address these issues.
%
We establish the insufficiency of these approaches, but also their commonality.
In particular, the existent solutions seek ad-hoc extensions to Tor, either
formally, via revisions to the protocol specifications, or externally, through
cooperating services that interact with Tor nodes.


We then propose our thesis: namely, that it is possible to build a generic extension
mechanism into Tor that allows for increased onion service availability,
performance, privacy guarantees, and usability, without reducing the privacy
properties of the the system vis-a-vis the current Tor ecosystem (unless the
onion service expressly chooses such a trado-off).
%
In this document, rather than propose a specific solution, we take the first
steps to explore the design space, identify the challenges, arrive at the
desired properties, and compare our approach to the existent solutions.



\subsection{Overview}
% describe the system and the core problem as it relates to this system


\subsection{Related Work}
% describe any related work that has tried to solve this problem and the
% tradeoffs of those solutions

\subsection{Availability}

\subsubsection{Service Scaling}

In Tor, there is a tussle between service scaling and service anonymity:
it is difficult to strengthen one without weakening the other.
%
This was not the intent in the original design of Tor, but rather evolved from
numerous deanonymization attacks that highlighted the need to restrict an onion
service's path selection.
%
In other words, while an operator can spin up multiple replicas, for
anonymity-sake, the replicas are contrained to use the same small set of paths;
thus replication is illusory, as it can only scale to the max bandwidth of the
shared paths, rather than the bandwidth of the network.
%
We briefly describe these attacks.


A major privacy consideration in Tor is a client or onion service's
selection of the first relay in its circuit: if an adversary can observe the
edges of a circuit, then, based on traffic signatures, she can confirm
who is communicating.  
%
In the case of onion services, an attacker can trivially be both the client
and rendezvous point, and thus observe these portions of the circuit.
%
This means that the service is susceptible to de-anonymization in \emph{every}
communication path: the attacker simply must determine that one of her relay's
is on the service's half of the rendezvous circuit, and that this relay is the
service's first hop.

% cite: [06-oakland] locating hidden server 
Outlines four low-resource techniques for an attacker to
determine whether her relay is the onion service's entry node.
%
The paper then outlines a probabilistic defense (which later became a default
Tor practice)  in which the onion service pins its entry node to a particular
relay (now called the \emph{Guard} node), rather than choose an entry node
at random for each new circuit.
% TODO: can cite guard-spec.txt


However, these attacks, as well as others (trolling) still allow an attacker to
discover the Guard node.
%
The concern is then that a Guard discovery attack, combined with a comprosimie
and/or coercion of the Guard node, may lead to the deanonymiation of the onion
service.


Proposal 292 proposes and implements (partly in Tor, partly via Stem) an option
for an onion service to also pin its second and third hops to a a small set of 
relays.
%
The result is a small tree of relays, termed a vanguard mesh, through which the
onion service constructs its circuits.
%
With this new path selection, an attakcer is forced to composomise one or more
nodes before learnign the guard node of an ionion service.
%
Additionally, to avoid linkability, the proposal also inserts an extra middle
node after the thrid layer guard/node for client-side intro and hsdir ciruits,
and service-side rendezvous circuits.

%
% TODO: An Anonymity System for caching content
% Tor doesn't do this.  One sentence on:
%   However, the idea of hiding the location of a document (or
%   encyrpted fragment of a document), perhaps within the peer-to-peer network,
%   underlies many censorship-resistent publishing designs such as Freenet,
%   Free Haven, and Tangler.
%
% TODO: A few sentences on that POC of running a DHT over Tor and the issues with
% this.


\subsubsection{Load Balancing}
% Desirable Goals:
%   - Obscure the number of hidden service instances.
%
%   - No master hidden service instance
%
%   - If there is a master hidden service instance, have clean
%      fail-over from one master to the next, undetectable by the network.
%
%   - Obscure which instances are up and which are down.

While a hidden service may run an application-level load balancer (for
instance, HAProxy) to dispatch connections and application-level
requests to various backend replicas,  such a solution places the load-balancer
on the critical path, and thus positions it as a potential bottleneck.  
%
As a result, we are interested in load-balancing solutions that redirect a
client to a different service replica \emph{before} the client even connects to
the service.
%
Normal Internet services achieve this form of load-balancing by interposing
at the DNS and routing level.
%
The loose analogues in the Tor network are to interpose at the HSDir and
Introduction-level, respectively. 
%
We describe the current approaches to load-balancing at each of these levels.

\parhead{HSDir Level}

\begin{widelist}
\item \textbf{OnionBalance}
% TODO: need to cite how much load-balancing this achieves.
% cite: https://onionbalance.readthedocs.io/en/latest/design.html
is a service external to Tor.  The key idea is that an onion service operator
publishes a different descriptor to the HSDirs for each replica; each
descriptor specifies a different set of Introduction Points.
% 
The operator then runs OnionBalance, which exhaustively looks up the descriptor
for each replica, and publishes a new descriptor (under the the service's
well-known \texttt{.onion} name) that includes a sample of the replicas'
Introduction Points (recall that V2 descriptors are limited to 10
Introduction Points, and V3 to 20).
%
OnionBalance also includes a mode whereby replicas \emph{push} their
Introduction Points to the OnionBalance process over a secure channel,
which prevents certain enumeration attacks by a malicious HSDir in the case
of V2 descriptors.
%
Since the default Tor client randomly chooses an Introduction Point from those
in the descriptor; OnionBalance distributes clients evenly across the
different replicas.

% issues:
%   - not very adaptive (slow failover, etc.)
%   - number of advertised introduction points is limted
%   - the secure channel mode is unspecified, would likely occur over a Tor
%     hidden service for the OnionBalance process, and thus makes this
%     hidden service a single point of failure for the replicas, leading
%     to infinite recursion (turtles all the way down)


\item \textbf{Multi descriptor publishing}
% cite [15-theses] Tor: Hidden Service Scaling

Recall that an onion service publishes its descriptor to multiple HSDirs
(historically six, now eight), both as a means of handling churn in the HSDirs,
as well as to spread the load of HSDir lookups.
%
The idea of multi-descriptor publishing is for an onion service to provide
each of these HSDirs with a descriptor for the \texttt{.onion} address, but
with a set of different Introduction Points.
%
This approach is complimentary to the use of OnionBalance.
\end{widelist}


%%%%%%%%%%%%%%%%%%%%%%%%%%%%%%%%%%%%%%%%%%%
\parhead{Introduction Level}

\begin{widelist}

\item \textbf{Prop-255}
% cite: proposals/255-hs-load-balancing.txt 
decouples \emph{introduction} from \emph{rendezvous},
by allowing one replica to service the introduction, and another to create the
rendezvous circuit.
%
The proposal outlines an extension to Tor's control specification to allow a
replica to process the introduction request and hand off the remaining steps
to an external process; the external process then dispatches another
replica with the necessary state to form the rendezvous circuit.


\item \textbf{Load-Balancing at the Introduction Point}
% https://lists.torproject.org/pipermail/tor-dev/2013-October/005556.html
% http://tor.stackexchange.com/questions/13/can-a-hidden-service-be-hosted-by-multiple-instances-of-tor/24#24
seeks an extension to Tor whereby multiple onion service replicas maintain a
long-term connection to an Introduction Point; the Introduction Point then
schedules (for instance, round-robins) introduction requests for that service
among the set of connected replicas.

% TODO: What is revealed by having HS replicas build circuits to common Introduction
% Points?

\end{widelist}

\subsubsection{DoS Defense}

One of Tor's original goal was the condition that taking down an onion service
implied taking down the entire Tor network (TODO: need cite, probably 04-sec).
%
Numerous attacks that targeted only a small number of relays belied this goal.


\parhead{Valets Services}
% cite [06-pets] Valet Services: Improving Hidden Servers with a Personal Touch
% cite [07-pets] Improving efficiency and simplcity of Tor circuit
%                establishment and hidden services
Valet services is a design proposal for mitigating denial of service attacks
against the Introduction Points.


A Valet is a normal Tor relay that sits in front of an onion service's
Introduction Point (from the perspective of a client).
%
Valets hide the Introduction Points from the clients, and, in turn, Valets do
not directly know the onion service that they protect.


The Valet design assumes the existence of a directory service (something like the
DHT for HSDirs) to which onion services publish tickets and clients retrieve
tickets, as well as a mechanism for an onion service to provision the
Introduction Points with necessary information to validate tickets.
%
The tickets themselves contain one-use tokens, encrypted with a Valet node's
public key, that specify that the Valet may connect to a certain Introduction
Point as part of a client's introduction to given onion service: the onion service
is masked to the Valet but known to the Introduction Point.


Valets chiefly change the topology of an onion service's contact points from a
flat array of Introduction Points to a two-level tree  where the Valet level
can contain many more nodes than the Introduction Point level.
%
In this way, it is more difficult for an attacker to focus a DoS attack on a
specific Introduction Point.


\parhead{Proposal 305}

Proposal 305 proposes an extension to the Tor relay's denial-of-service
subsystem so that an onion service may specify to an Introduction Point the
limits (via an extension to the \texttt{ESTABLISH\_INRO} cell) on the rate of
introductions the Introduction Point relays to the onion service, rather than
use the limits specified in the network-wide consensus parameters.
%
The proposal also suggests a new cell type that the Introduction Point sends
back to the onion service when the rate limit is exceeded, so that the onion
service may decide how to respond to the event.


\parhead{Onion Service Authentication}

% TODO: there is authentication baked into multiple levels of the onion service
% spec.  Certainly one of the purposes of this is to prevent DoS by delaying an
% onion service to actually commiting resources until the client seems legit.
% IIRC, very few services use this.


\subsection{Performance}
% TODO:
%   obviously, availability is related to performance; so here, we're concerned
%   with the fact that onion services are slow, and the hypothesis is that
%   they're slow because the circuit is longer than normal circuit.
%
%   TODO: talk about work that tries to make the circuit shorter, as by merging
%   the Introduction Point with RP.
%
%   TODO: talk about different onion flavors: single-hop onions and exit
%   enclaves
%
%
%   [07-pets] improving efficiency and simplicity of Tor circuit establishment
%             and hidden services
%
%
%   [07-thesis] performacne evalution of tor hidden services
%
%   [08-saint] performance-metrics-and-statistics-of-hidden-services
%       read, not much there other than some discussion of cannabilization
%       the first author of this and the 07-thesis paper are the same
%       [07-pets] is the biggest prior work
%
%   [08-thesis] improving tor hidden service protocol 
%       this is a coauthro of the 08-saint paper
%       good description of tarzan, i2p, p5
%       I think the point of this paper is that it implements [07-pets]
%       and evaluates the performance after these modifiations.
%
%
%   [13-thesis] performance of hidden services in tor
%   [15-techreport] hidden service stats techreport
%   [16-unknown] performance-and-security-improvements-for-tor
%

\subsection{Privacy Guarantees}
% TODO: talk about geographic and ASN routing, and how its hard to extend to
%       onion services.  The general problem is that there are all different
%       flavors of relay selection algorithms for clients, but these don't
%       translate to Introduction Point and RP circuits, since half of the
%       circuit is not under the client's control.
%
% TODO: might be pedantic (I don't know all of the details) but it relates to
%       what information an onion service leaks (by necessity) to the Tor
%       network, and to which nodes in the network.
%
%   [06-blackhat] locating hidden services
%   [11-wifs] fingerpritning tors hidden service logs using timing side channel
%   [15-sec] circuit fingerprinting attacks deanonymize hidden services
%   [15-ccs] caronte-detecting locaitno leaks for deanonymizing hidden services
%   [16-sec] identifying and characterizing sybils in the tor network
%   [16-ccs] poster fingerpritning tor hidden services
%   [17-wpes] analysis of fingerprinting techniques for tor hidden services

\subsection{Usability}
% TODO:
%   The biggest usabilty challenges are
%       - human-readable names for onion addresses,
%       - onion service discoverability (if desired)
%           - no search engine or indexing is currently available
%       - composing onion addresses with HTTPS certs


\subsection{Prior or Proposed Work}
% describe any work you've already done in this area, and any new
% work you propose to do.  For new work, re-emphasize the problem you are
% trying to sovle, sketch a direction of the work, and describe how you will
% evaluate if the work is successful.
