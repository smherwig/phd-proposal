\section{Practical Information Flow Tracking}

% Possible sections/components: 
% hypervisor and qemu background, filesystem
% taint label definition "calculus"
% distributed dift
% Taint explosion and over-tainting
%
% What is the high-level problem
%   Same as keyless
%
% What are the goals
%   - if you're just protecting the private key, it's lame.  Try
%     to ensure integrity or confidentiality of more things.
%
% What is the motivating use-case
%   - CDN
%
% How is taint represented and stored; what do labels mean?
%
% How does the transition from protected VM to monitor work (V2E)?
%   How does the monitor ensure the transition is valid (i.e, CFI stuff).
%
% How does the transition form monitor VM to protected work (E2V)?
%
% How is taint persisted on a filesystem
%
% How is taint initialized?
%
% How is taint propagated across a network?
%
% How does taint deal with multiple principles/tenants?
% 
% How does taint migrate during a VM migration?
%
% How are VMs provisioned? (by customer or CDN operator, in the case of renting)
%
% Can taint-based deadlock occur?  How is this resolved?
%
% Can taint be declassified?

\subsection{Problem}
Central problem: controlling the flow of sensitive user data in a distributed
environment without modifying the software stack or augmenting the hardware
platform.


\subsection{Abstract Ideas}

Fine-grained information flow tracking (IFT) is conceptually simple: the
runtime environment tags memory and registers containing sensitive data with
labels, and propagates these labels through the memory space and filesystem in
acccordance with the computation.
%
% (sources and sinks)
Traditional use-cases for IFT have a source and sink model where labels are
assigned at sinks and checked at sources.  For instance, untrusted data recieved from the
network might be tagged, and the code that constructs a database query checks
that the query does not containted tagged data.  Similary, the runtime might
check that tagged data has not overwritten a control register (i.e.,
\texttt{RIP}), which might be indicated of an exploit attempt.

IFT is also used as a sort of mandatory access control on data and derived data,
and thus restirctsion the flow of information between users in an
organization; security checks need to be performed only when the data is
externalized in some fashion and no limitations need to be imposed on how
the data is handled locally on a user's machine.  


\subsection{Trust Model}
For our trust model, I assume that hosts faithfully execute the application
code, but may abuse the access controls and execution transfer to attempt to
glean private data.


Because we're not currently tracking implicit flows, it would be trivial to
leak secrets for a malicious application.  
% not sure I agree with this
%
It would also be possible to hijack the declassification sequence to declassify
data.   We trust the untrusted machine to not modify the application of
migration system and run the correct code.


\subsection{Design}

PIFT (Practical Information Flow Tracking) is a full-system IFT plaftorm.

The hypervisor interposes itself between the hardware platform and the set of
virtual machines, mediating all access to the physical resources, as well as
all inter-VM communication.
%
Hypervisors routinely virtualize and mediate access to exit points (i.e.,
output devices such as network interfaces and blcok storage devices).
%
The typical implementation approach for full-system IFT involves running the
guest system in a hardware emulator (e.g., QEMU), that has been augmenteed
with machine instruction analysis and taint tracking capabilities.

%
Hence, PFIT intercepts device I/O requests that externalize data in order to
encforce policies, but does not try to prevent applciation code from touching
or performing computaiton on sensitive data.


Our implementation employes on-demand taint tracking --- a technique whereby
the guest system is dyanmically transferred between emulated execution using
QMEU and native execution within a Xen guest domain. Enabling emulation and IFT
compution only for those regions of guest code thta direclty interact with
tainted data allows PIFT to substantially reduce the runtime performance costs.

The control VM has a list of valid entry points that may be transferred to and
uses it's own code segment (rather than relying on the code segment being
transferred over).



\subsection{Labels}
PIFT's mechanismes for policy specification and enforcement are based on the
decentralized label modes [56, 57] -- a simple, but powerful model of access
ctonrol that enables multiple principals to protect their proivate information
and share it in a controlled manner.  In a decentalrized label model, each data
value is assigned a alble, which expresses a certain set of restrctiions on its
disemination.  Conceptually, a label represents an unordered set of
confidentialiaty policies.  Each policy has an owner and defines a set of
authorized readers.  


\subsection{Design}
The focal component of our design is an augmented hypervisor -- a thin software
layer that exposes the underlying hardware platfrom in a virtualized form and
allows several virtual machiens to execute concurrenlty.  Our system is based
on Xen.

All user-facing applications run inside a protected virtual machine (VM) atop
the hypervisor.  In addition, a special control VM operates in the background
and provides a number of supporting modules: a robuest full-sytem emulator, a
label-aware filesystem, and drivers for virtualized I/P devices.  PIFT-XEN
tracks the propagation of labels between the virtual CPU registers, memory, and
disk belonging to the protected VM.  The hyperviros also intercepts all
externally observed output actions (e.g., network communications, writing
data to a mobile storage decie, sending  data to a printer) that enforces
policies allowing or denying specific application requests to externalize
sensitive data.

Initially, all application- and OS-level code in the protected VM execute at
native speed directly on the host CPU, but the hypervisor instruments the
hardware page tables in a manner that allows us to intercept instructions that
access tainted memroy pages.  When the application first opens the sensitive file,
the call is routed via the hypervisor to the label-aware filesystem running
in the control VM.  Before returning the file handle, the filesytem makes a
call to the hypervisor, informing it of the file's label.  In response, the
hypervisor marks the memory pages holding the file contents as tainted and
updates the page tables accordingly.

When an applciation running in the protected VM tries to access the file
contents from a tainted memory page, the hardware memory management unit
generates a page fault and immediately transfers control to PIFT-Xen.
The hypervisor suspends the native exeuction context of the protected VM, takes
a snapshot of its CPU registers state, and transfers contorl to our augmented
emualtor, which resumtes tje execution of the proteced VM in emulated mode.
Our current implementation handles emualtion using a heavily-modified verison
of QEMU, which runs as a uer-level process in the control VM.  The emulator is
instrumented to tack the movment of labels in accordagne with the IFT model.

On a conceptual level, PIFT associates a data label with each
individually-addressable byte in the protected VM.


\subsubsection{Distributed Labels and Externalizing Tainted Data}
In order to ensure that taint labels are preserved across network transfers,
PIFT transparently augments the networking stack in the guest environment to
intercept all outbound network packets and annotate the payload with the
corresponding taint labels.
%
When the protected VM tries to externalize tainted data through an I/O device,
PIFT intercepts the device request and invokes the appropraite security
checks.  These operations are handled by a group of device-specific
interception modules that are implemented as extensions to the backend
device drivers operating in the control VM.


\subsubsection{Label-aware Filesystem}
PIFT provides a taint-aware filesytsem that enables users to store sensitive data
(annotated with appropriate taint labels) persistently on disk.
In the current desing, we create and deploy a PIFT-ext3 filesytem inside the

control VM and export it to the user-facing VM through a standard remote file
access protocol (NFSv3).  While this split-up configuration is not as efficient
as running PIFT-ext3 directly inside the protected VM and incurs addtional
latency overhead, it allows us to mainting full compatibilty with unmodified OS
biaries.


\subsection{Challenges}

\subsubsection{Taint Explosion}
For kernel taint explosion, we show that a major cause of this phenonmenon on
Linux is accidentail tainting of kernel control data structures.  We undertake
a detailed analysis and track its origin to a small number of kernel entry
functions.  By interposing at these specific entry poitns and securtly
scrubbing tant, we prevent accidetnail taining of these data strucutres and
effectively eliminate Linux kernel tain explosion for all practical purposes.

\subsubsection{Liveness}
\paragraph{Declassification}
%
A tainted private key will result in a tainted signature.
%
Similarly, a tainted SQLite file will result in tainted query result.
%
Thus, there needs to be a gate that untaints the data, and such a gate almost
certainly needs to be externally specified (as by the policy itself) (as by the
policy itself).  
% 
Thus, the policy still boils down to knowing the sinks (the sybmol names) that
declassify/untaint data.


\paragraph{Taint Deadlock and Thrashing}:
%
Problems will arise when we have tainted and non-tainted data on the same page.
%
Specifically, if an untrusted machine needs data on a page with tainted data,
right now it won't migrate the page.  
%

The proposed approach to avoid this situation as much as possible is to use two
memory allocators---one private and one public; trigger migration on entrance
to "private library" (a library we expect to allocate memory that will become
tainted, eg openssl).  
% 
When calling malloc from a private library, trigger migration private malloc
only on private machine private allocator allocates to separate set of pages
allocators for each type of taint - just map to separate area of memory; malloc
takes heap as argument allocator for each lattice level--- machines should
access allocators up to but not above their level.


However this doesn't guarantee the situation will never happen.
%
A fall back could be to zero out tainted data on pages, however the application
isn't aware of this so need to ensure program correctness.
%
On page fault, check addr is not tainted, then only load page for as long as is
needed to load that data, then remove this probably isn't possible at page
fault level, only have page address or run pin on all machines to do memory
tracking for tainted data that has been zeroed out---would require checking on
each memory access




\subsection{Implementation Details}
Intercepting the Initiail Access to Sensitive Data:  The key challenge in
trapping tainted data access is efficiency: a naive implementation would trap to
the hypervisor upon every memory access from the guest VM to determine whether
sensitive data is being accessed, but this strategy would incur unacceptable
overhead.  Instead, we leverage the capabilities of the hardware paging unit
and configure it to generate a trap upon every access to a memory page that is
known to contain sensitive data.  To accopmlish this, PIFT-Xen creates a set of
shadow page tables for the guest environment, clearing the PAGE\_PRESENT (P)
flag in the shadow PTEs that hold mapping to tainted memory pages.  Thus, when
the guest VM tries to access a tainted page (either for reading or writing),
the memory managment unit gernates a  page fault and transfers control to a
hypervisor-level fault handler.


Switching between Virtualized and Emulated Execution: Migrating the protected
system from the virtualized mode of exeution to the emulated mode involves
suspending the native VM, producing a comprehensive snapshot of its virtual CPU
state, and initializing the emulated processor from this snapshot.  In PIFT,
the
hypervisor and the emulator coordinate their activities and exchange state using
a common data structure in a shared memory page.  in the first phase, the
hypervisor initailzies the guests' strcuture that encpasulstes a snaphsot of
the virtual CPU.  Once a comprehensive snapshot has been obtinated, teh
hypervisor signals QEMU through a virutal IRQ and instructs it to initiaate
emualtion.  When QEMU receives this signal, it initailizes the state of the
emulated CPU based on the information containted in the snapshot and luanches
teh main emulation loop.


Our approach to taint tracking is based on augmenting the emualted machine with
a virtual hardware extenions in the form of a taint processor.

The emulated (guest) machine executes in the context of a single user-space
emulator process in the host OS and all elements of the guest machine state,
including its CPU registers, physical memory, and various peripherals,
are represetned with corresponding data structures in the address
space of this process.  To emulate memory accesses from the guest machine, QEMU
implements a software-based memory management unit that provides guest-virtual
to guest-physical address translation, mimicking the behavior of a hardware MMU
for the guest architecture.  In addition, a software-based TLB maintains
mapping between guest virtual addresses and the corresponding addresses in the
emulator's own virtual address space.

The taint processor is the central arehtictural module of our information
flow tracking substrate.  This module is tasked with consuming and executing
blocks of taint tracking insturctions produced by the code translator.


In order to realize the full benefits of PIFT, including its taint-aware
storage and on-demand emulation capablities, we must establish proper
interactions between the emualtor and the other central components of the
system, such as the hypervisor adn the taint-aware filesystem.  This
necessiates several addtional changes and extensions to QMEU.

% During statrtup, the emualtor allocates ...




\subsection{Reasons why CDN Taint Tracks}
% The thought is that the CDN might want to taint-track for the reasons these
% taint-trackign systems were originally developed, and our solution should
% compose with this usage as well.

%I'm also wondering how this jives with traditional taint tracking, the canonical
%example of which is tainting data read from a network socket, and
%checking that the data is not passed thorugh to a system() or a database call
%without first being untainted by some sanitization function.
%%
%This is essentially saying that untrusted network data is exclusively pinned
%to an isolation domain.


\subsection{Prior Work}
At a conceptual level, the design of PIFT bears some similarliy to Loki, which
pushes the data labeling functionality into the hardware platform.

Yet another influential and related research direction is centered aroudn the
Red/Green isolation paradigm articulared by Butler Lampson et al.

While in TaintCheck and Sweeper policies impose restrictions on fine-grained
data movement within a process (e.g, "the instruction poitner cannot be loaded
with a tainted value") with the overal goal of enforcing system
integrity, PIFT policies seek to ensure confidentiality by confining the flow
of sensitve user data between principals.

Neon explores system support for derived data managment and proposes a set of
mechanisms that enable orgainzation to enforce end-to-end data containment
policies.  Analogously to PIFT, this system seeks to achive binary-level
compatibility with exiting operating systems and paplications using a
hypervisor-based desing.  Neon asscoiates a 32-bit tint, which represetns a
policy, with each byte in the guest virtual machine and uses a modified version
of QEMU to track the propagation of tints between memory and CPU regsiters at
the level of machine instructioms.  Similar to its predecssor and our system,
Neon implements on-demand emaultions and uses the paging hardware to trap
the inital access to senstive data during native execution.

Ho el al. [41] propose demand emulation as a practical technique for online
full-system taint tracking, whereby a hypervisor and taint tracking emulator
cooperate to dynamically switch the guest system between native virtualized and
emulated modes of exeuction.




