\section{Background}
\label{sec:background}

In this section, I clarify the problem with specific examples, and specifically
focuse on the everday case of outsourcing.
%
%
In this setting there are two principals: the \emph{organization} and the
\emph{cloud service provider}.
%
I describe the security and privacy implications of naively outsourcing, as
well as several trust models that apply to this setting, 
%
I then enumerate a list of goals for a system designed to handled these
implications.


\subsection{Problem}

There are two broad approaches of outsourcing.
%
The first, \emph{cloud hosting}, is an organization migrating its in-house
services to a third-party deployment; the second is an organization  migrating
from in-house services to those offered by third party \emph{cloud services}.
%
Hybrid approaches exist, such as where a cloud service provider must
communicate with the organization's, in-house, backend services.
%
Common outsourcing examples are an organization moving its HTTPS, DNS, or
email servers into the cloud, or contracting some other party to run its own
implementations of these services on the organization's behalf.
%
Nowadays, all of these core application-level protocols use public key
cryptography to ensure the integrity, authenticity, and/or confidentiality of
the application's content.


As a concrete example, consider an organization that outsources its web hosting
to a a content delivery network (CDN)\@.
%
A CDN is a massive, global network of \emph{edge servers}, where each edge
server acts as a reverse proxy for the organization: to
handle client requests, edge servers retrieve content from the organization's
\emph{origin servers}, and cache it so they can deliver it locally.
%
CDNs derive much of their utility from the fact that they have servers close to
most clients, thereby providing low-latency responses.
%
Several techniques exist for routing clients to the closest edge server, but
DNS is common, and so the CDN may also handle the organization's DNS server.
%
CDNs also provide added security benefits, such as absorbing DDoS
traffic, and filtering targeted malicious trafic, such as SQL injection and
cross-site scripting attacks~\cite{securing-cdns}.
%
Virtually all of the most popular websites (and a very long tail of unpopular
websites) use one or more CDNs to help reliably host their content.


With these use-cases in mind, let's enumerate the security and privacy
implications of outsourcing such service such services.


% XXX: try to keep each implication to one meaty paragraph.
\parhead{Impersonation}
% byzantine
%
In order for the provider to fulfill the service, the provider must
have access to the application's private keys.
%
Indeed, as the web moves towards HTTPS-everywhere~\cite{felt-2017-https},
organizations increasingly rely on CDN providers to store their HTTPS
certificate and the corresponding secret keys~\cite{key-sharing,
when-https-meets-cdn}, so that they can accept TLS connections while
maintaining low latency to clients.
%
Even with an application like DNS, where an organization would traditionally sign
its DNS records offline, the organization may choose to outsource the keys
and management to the provider.
%
% TODO: cite cloudflare blog
With the organization's private DNS keys online, the provider can offer
additional services, such as greater flexibility to return geographically-based
responses, as well as zone enumeration defenses.
%
This has significant implications on the trust model of each PKI and the web
writ large: today's providers can arbitrarily impersonate any of their customer
organizations.


\parhead{Breach of Confidentiality}
% Honest but curious
%
With the provider acting as a man-in-the-middle between the
organization and its clients, the provider is privy to the client's private
data, such as passwords, cookies, and personal content, as well as metadata 
for client profiling.
%
The provider might further subcontract aspects of the deployment (as with an
email provider using another provider for virus and spam scanning), with
the client and organization unaware of the extent of their data exposure.


\parhead{Correctness}
% concern 4: correctess: e.g., DNSSEC providers not actually providing DNSSEC.
% Negligent (also, a supertet of the byzantine and honest but curious)
%
By outsourcing a service, the organization no longer has guarantees that the
provider adheres to the expected service.
%
For instance, one DNS study~\cite{chung-2017-dnssec} discovered that 31\% of
domains that claim to support DNSSEC fail to publish all relevant records
required for a client resolver to validate the response, while
a study~\cite{open-http-proxies} of open HTTP proxies found that 5\% perform
unwanted or malicious content modification.


% TODO: call this something else
\parhead{Latent Trust Assumptions}
%
Assume a hybrid deployment where the organization has migrated its front-end
business applications to a cloud service provider, but that the application's
database remains within the organization’s network. 
%
Before migrating to the cloud, the database could trust that any connections to
it were from benign, internal tools; in the new deployment setting, the
organization must guard against potentially malicious or devious queries from
the cloud front-end.


\parhead{Legal and Regulatory Restrictions}
% concern 5: legal, policy compliance, ...restrictions
% talk about akamai not hosting content on thrid parties.
% talk about HIPPA (as with emails)
%
By outsourcing a service, the organization may be non-compliant with
industry regulations.
%
For instance, a hospital may be unable to run its electronic medical records
system in the cloud due to concerns  of HIPPA compliance.
%
In other words, even if the organization itself accepts the security and
privacy implications of outsourcing their services, legal matters may restrict
the degree to which the organization can leverage such deployments.


\subsection{Trust Model}

\begin{table}[t]
%\resizebox{.5\textwidth}{!}{
\small
\centering
\rowcolors{2}{gray!15}{white}
    \begin{tabular}{@{}lcccl@{}}
        & \textbf{Enclave}& \textbf{Provider} & \textbf{Software} & \textbf{Deployment} \\
        \hline
        1 & \cmark          &                   & \cmark          & Conclaves   \\
        2 &                 &                   & \cmark          & Gemini      \\
        3 & \cmark          &                   &                 & out of luck \\ % sandbox in enclave
        4 &                 &                   &                 & out of luck \\ % sandbox in partition
\end{tabular}
%}
\caption{The deployment strategy, depending on whether the provider offers
    trusted enclaves, whether the organization trusts the service provider, and
    whether the organization trusts the provider's application software.
    }
\label{tab:trust-models}
\end{table}

We describe an idealized trust model from the perspective of an
organization that uses a cloud service provider that requires the organization's
sensitive data.
%
For simplicity, we assume that the provider and the cloud machine owner are the
same party, and that the provider administers all software on the machine,
including any system software, such as the operating system, as well as the
application.
%
An untrusted provider may deviate arbitrarily from the stated service, but
remains subject to standard cryptographic assumptions.
%
Likewise, untrusted software may deviate abitrarily from its stated purpose
and, in particular, may actively try to leak sensitive data.
%
We assume, however, that the software is bug-free, and thus the model omits
external actors, such as a remote attacker.
%
The application may be composed entirely of open-source components, private
components that are proprietary to the provider, or some combination thereof.
%
If trusted boot is available  (either via a TPM, or in the form of secure
enclave launch), we assume that the organization and provider can agree to run
a specific build of system software, and that such software is trusted and
bug-free.


Table~\ref{tab:trust-models}, shows the different possible trust models based
on whether the provider offers hardware enclaves, whether the organization
trusts the provider, and whether the organization trusts the provider's
software.
%
We omit the uninteresting cases with a trusted provider, with the assumption
that the organization transitively trusts applications that the provider
offers.


In model 1, the organization trusts the software, but does not trust the
provider to run it faithfully, and thus uses one of various
approaches~\cite{talos,haven,scone,graphene} to
run an unmodified legacy application in an enclave.
%
This is the trust model for my prior work, conclaves, described in
\S\ref{sec:conclaves-summary}.
%
In contrast, in model 2, an enclave is not available (or assumed vulnerable to
to side-channnel attacks).
%
This is the starting use-case for Gemini, described in \S\ref{sec:gemini-design}.
%
In models 3 and 4, where neither the provider nor the software is trusted, no
general purpose solution exists for an organization to securely use an
commodity applicatins.


%For instance, the application may be composed of several components, some of
%which are trusted, and some are not; we assume that the organization
%trusts the components that interact with the senstive data.

In practice, the model's binary characterization of ``trusted" is too coarse,
the definition of untrusted as ``arbitrarily malicious" too strong,
and the assumptions of bug-free code and attacker-free environments too
naive.
%
For instance, it may be the case that most of the application is untrusted, but
the organization trusts a small portion of the application, such as its
cryptographic library.
%
This scenario is between models 2 and models 3 and 4, as the organization can
identify a small trusted computing base within the application.
%
Moreover, in the presence of attackers, an organization may simply
want strong, physical, isolation of its sensitive data from a remote attacker.
%
In other words, trust is a spectrum; the model highlights the endpoints of the
spectrums; my systems (conclaves and Gemini) target a specific endpoint, but,
to varying degress, allow for the relaxation of assumptions.



\subsection{Goals}

Our high-level goal is to allow the user to specify a trust policy to the
runtime execution environment; the runtime transparently implements the
policy on behalf of the applications.
%
I break down the policies into two categories:

\begin{widelist}
\item \textbf{Confidentiality policies:} Specifications of isolation boundaries for
    sensitive resources; the run-time constructs and enforces the boundaries.

\item \textbf{Integrity policies:} Specifications for data integrity; the run-time
    provides mechanisms for measuring integrity and detects breachs of
    integrity.
\end{widelist}


Our primary constraint is that the runtime should not require applications to be
rewritten or recompiled.
%
This is a practical constraint for system adoption, but also underscores our
thesis that such policies are deployment-specific, may be unknown during the
application's development, and must handle ``black box" applications, where the
source code is unavailable.
%
Our secondary constraint is that the policies should be simple; I want to avoid
degenerate solutions where the specification of the policy is more
onerous than simply rewriting the application.

Additional practical constraints, with respect to outsourcing, is that the policies
and runtime should handle applications that multiplex services across organizations,
with strong isolation between the organizations.
%
The system should also avoid degenerate solutions
that transfer undue hosting responsbilities back to the organization, thereby
defeating the intent of outsourcing.


Intuitively, the system will have worse performance than its vanilla
counterpart.
%
Nevertheless, the system should allow introspection into performance bottlenecks
as a means of investigating possible optimizations.


As non-goals, we are not concerned with obliviousness of the access paterns,
which may be a side-channel vector.
