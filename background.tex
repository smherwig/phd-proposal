\section{Background}
\label{sec:background}

In this section, I clarify the problem with specific examples.
%
While the abstract problem could apply to several concrete ones, we focus on
the everyday case of outsourcings.
%
Here there are two principals, the organization and the cloud service provider.
%
I describe the security and privacy implications.
%
Several trust models could apply to this situation, and I describe the them.
%
I then enumerate a list of goals for a system designed to handled this
situation.


\subsection{Problem}

There are two broad flavors of outsourcing.
%
The first, \emph{cloud hosting}, is an organization migrating its in-house
services to a third-party deployment; the second is an organization  migrating
from in-house services to those offered by third party \emph{cloud services}.
%
Hybrids exist between these two extremes, such as where a cloud service
provider must communicate with the organization's, in-house, backend services.
%
Common outsourcing examples are an organization moving its HTTPS, DNS, or
email servers into the cloud, or contracting some other party to run its own
implementations of these services on the organization's behalf.
%
Nowadays, all of these core application-level protocols use public key
cryptography to ensure the integrity, authenticity, and/or confidentiality of
the application's content.


As a concrete example, consider an organization that outsources its web hosting
to a a content delivery network (CDN)\@.
%
A CDN is a massive, global network of \emph{edge servers}, where each edge
server acts as a reverse proxy for the organization: to
handle client requests, edge servers retrieve content from the organization's
\emph{origin servers}, and cache it so they can deliver it locally.
%
CDNs derive much of their utility from the fact that they have servers close to
most clients, thereby providing low-latency responses.
%
Several techniques exist for routing clients to the closest edge server, but
DNS is common, and so the CDN may also handle the organization's DNS server.
%
CDNs also provide added security benefits, such as absorbing DDoS
traffic, and filtering targeted malicious trafic, such as SQL injection and
cross-site scripting attacks~\cite{securing-cdns}.
%
Virtually all of the most popular websites (and a very long tail of unpopular
websites) use one or more CDNs to help reliably host their content.


With these use-cases in mind, let's enumerate the security and privacy
implications of outsourcing such service such services.


% XXX: try to keep each implication to one meaty paragraph.
\parhead{Impersonation}
% byzantine
%
In order for the provider to fulfill the service, the provider must
have access to the application's private keys.
%
Indeed, as the web moves towards HTTPS-everywhere~\cite{felt-2017-https},
organizations increasingly rely on CDN providers to store their HTTPS
certificate and the corresponding secret keys~\cite{key-sharing,
when-https-meets-cdn}, so that they can accept TLS connections while
maintaining low latency to clients.
%
Even with an application like DNS, where an organization would traditionally sign
its DNS records offline, the organization may choose to outsource the keys
and management to the provider.
%
% TODO: cite cloudflare blog
With the organization's private DNS keys online, the provider can offer
additional services, such as greater flexibility to return geographically-based
responses, as well as zone enumeration defenses.
%
This has significant implications on the trust model of the PKI and the web
writ large: today's providers can arbitrarily impersonate any of their customer
organizations.


\parhead{Breach of Confidentiality}
% Honest but curious
%
With the provider acting a man-in-the-middle between the
organization and its clients, the provider is privy to the client's private
data, such as passwords, cookies, and personal content, as well as statistics
for client profiling.
%
The provider might further subcontract aspects of the deployment (as with an
email provider using another provider for virus and spam scanning), with
the client and organization unaware of the extent of ther data exposure.


\parhead{Correctness}
% concern 4: correctess: e.g., DNSSEC providers not actually providing DNSSEC.
% Negligent (also, a supertet of the byzantine and honest but curious)
%
By outsourcing a service, the organization no longer has guarantees that the
provider adheres to the expected service.
%
For instance, in the case of DNS hosting, a prior study found that 31\% of
domains that claim to support DNSSEC fail to publish all relevant records
required for a client resolver to validate the response.


% TODO: call this something else
\parhead{Latent Trust Assumptions}
%
Assume a hybrid deployment where the organization has migrated its front-end
business applications to a cloud service provider, but that the application's
database remains within the organization’s network. 
%
Before migrating to the cloud, the database could trust that any connections to
it were from benign, internal tools; in the new deployment setting, the
organization must guard against potentially malicious or devious queries from
the cloud front-end.


\parhead{Legal and Regulatory Restrictions}
% concern 5: legal, policy compliance, ...restrictions
% talk about akamai not hosting content on thrid parties.
% talk about HIPPA (as with emails)
%
By outsourcing a service, the organization may be non-compliant with
industry regulations.
%
For instance, a hospital may be unable to run its electronic medical records
system in the cloud due to concerns  of HIPPA compliance.
%
In other words, even if the organization itself accepts the security and
privacy implications of outsourcing their services, legal matters may restrict
the degree to which the organization can leverage such deployments.


\subsection{Trust Model}

\begin{table}[t]
%\resizebox{.5\textwidth}{!}{
\small
\centering
\rowcolors{2}{gray!15}{white}
    \begin{tabular}{@{}lcccl@{}}
        & \textbf{Enclave}& \textbf{Provider} & \textbf{Software} & \textbf{Deployment} \\
        \hline
        1 & \cmark          & \cmark            & \cmark          & standard    \\
        2 &                 & \cmark            & \cmark          & standard    \\
        3 & \cmark          & \cmark            &                 & nuanced     \\
        4 &                 & \cmark            &                 & nuanced     \\
        5 & \cmark          &                   & \cmark          & Conclaves   \\
        6 &                 &                   & \cmark          & Gemini      \\
        7 & \cmark          &                   &                 & out of luck \\ % sandbox in enclave
        8 &                 &                   &                 & out of luck \\ % sandbox in partition
\end{tabular}
%}
\caption{The deployment strategy, depending on whether the provider offers
    trusted enclaves, whether the organization trusts the service provider, and
    whether the organization trusts the provider's application software.
    }
\label{tab:trust-models}
\end{table}

We describe an idealized trust model from the perspective of an
organization that uses a cloud service provider that requires the organization's
sensitive data.
%
For simplicity, we assume that the provider and the cloud machine owner are the
same party, and that the provider administers all software on the machine,
including any system software, such as the operating system, as well as the
application.
%
An untrusted provider may deviate arbitrarily from the stated service, but
remains subject to standard cryptographic assumptions.
%
Likewise, untrusted software may deviate abitrarily from its stated purpose
and, in particular, may actively try to leak sensitive data.
%
We assume, however, that the software is bug-free, and thus the model omits
external actors, such as a remote attacker.
%
% TODO: what about components that proprieatry to the organization?
The application may be composed entirely of open-source components, private
components that are proprietary to the provider, or some combination thereof.
%
If trusted boot is available  (either via a TPM, or in the form of secure
enclave launch), we assume that the organization and provider can agree to run
a specific build of system software, and that such software is trusted and
bug-free.


Table~\ref{tab:trust-models}, shows the different possible trust models based
on whether the provider offers hardware enclaves, whether the organization
trusts the provider, and whether the organization trusts the provider's
software.


\parhead{Models we ignore}
In models 1 and 2, the organization trusts both
the provider and the software, and thus the application may be run in a
conventional environment.
%
Models 3 and 4 are nuanced, as the organization trusts the provider, but not
the software; the software itself might be proprietary, or simply too large for
the organization to properly vet.
%
We choose not to focus on these models as they either disobey an assumption on
the transitivity of trust (that is, a trusted provider deploying a non-trusted,
proprietary, application), or speak to the case of running buggy software,
the latter of which is both outside of our trust model and mitigated with
existing sandbox techniques.


% In the figure, use colors to show the models you treat in the thesis
\parhead{Models we focus on}
In model 5, the organization trusts the software, but does not trust the
provider to run it faithfully, and thus uses one of various
approaches~\cite{talos,haven,scone,graphene} to
run an unmodified legacy application in an enclave.
%
In contrast, in model 6, an enclave is not available (or not trusted to be
immune to side-channnel attacks).
%
This is the starting use-case for Gemini.
%
In models 7 and 8, where neither the provider nor the software is trusted, no
general prupose solution exists for an organization to securely use an
unmodified, legacy, application.


%For instance, the application may be composed of several components, some of
%which are trusted, and some are not; we assume that the organization
%trusts the components that interact with the senstive data.

In practice, Gemini's trust model is more nuanced, and the stringent
black-or-white depiction of ``trusted" is too coarse.
%
For instance, it may be the case that most of the application is untrusted or
otherwise proprietary, but the organization trusts a small portion of the
application, such as its cryptographic library.
%
This scenario is between models 6 and models 7 and 8, as the organization can
identify a small trusted computing base within the application.
%
Gemini handles this scenario, with enclave availability and enclave trust
treated as configuration details.


\subsection{Goals}

Having reviewed the security implications of outsourcing services, as well as
the emergent trust models, I now describe a set of goals by which to evaluate
solutions.
%
The goals fit into two categories: \emph{safety goals}, which follow closely
from the security implications, and \emph{practicality goals}, which address
the flexibility of the solution to handle varied trust models and to support
off-the-shelf software.

\parhead{Safety Goals}
%
In the case of a server that uses TLS, safety goals increasingly diminish the
adversarial powers of the provider to that of an on-path TLS attacker.

\begin{enumerate}
    \item[S1] \textbf{Protect private keys:}
        %
        Do not expose the organization's private keys to the provider.
        %
        Given that the servicr will still need use of the key, this goal alone
        merely equates to the organization retaining the exclusive right to
        revoke the key.
    
    \item[S2] \textbf{Protect session keys:}
        % prevents low-resource attacks for leaking session key for decoding of the TLS traffic,
        % for either offline or online analysis; also, prevents collusion where
        % leaks to colluding party who then injects content.
        %
        Once a connection is established, do not expose the ephemeral session keys (nor
        the sensitive material for session resumption) to the provider.
        %
        Protecting the TLS session keys does not in and of itself imply that
        the provider does not obsersce the decrypted TLS streams; however, it does
        prevent low-resource collusion whereby the provider leaks the keys to
        an on-path party, who may then record and decrypt the streams or
        inject malicious content.

    \item[S3] \textbf{Protect TLS streams:}
        %
        The provider cannot observe the the plaintext of the TLS streams.
        %
        Note howerver, that the provider still manages the streams and has
        control over the responses.

    \item[S4] \textbf{Protect organization content:}
        %
        The provider cannot view or modify the organization's private content.

    \item[S5] \textbf{Ensure correctness:}
        %
        Ensure that the provider is running the actual service as expected.

\end{enumerate}


\parhead{Practicality Goals}
%
The practicality goals enumerate concrete features that are favorable to a
system's adoption.

\begin{enumerate}
    \item[P1] \textbf{Support legacy applications:} 
        %
        Support legacy binary applications; the system should not require that
        applications be rewritten or recompiled.

    \item[P2] \textbf{Maintain outsourcing:}
        %
        Maintain the property that the provider hosts the majority of the computation,
        and avoid degenerate solutions that transfer hosting responsbilities
        back to the organization, thereby defeating the intent of
        outsourcing.

    \item[P3] \textbf{Allow configurable deployment and trust models:}
        %
        Solutions that depend exclusively on hardware enclaves are additionally
        hampered by two realities.
        %
        First, while hardware enclaves have become more prevalant, they are by no means
        ubiquitous; for instance, neither Intel SGX nor AMD-SEV---the two enclave
        implementations for the x86\_64 architecture---are available on Amazon AWS\@.
        %
        Second, both implementations have, subsequent to their release, required
        patches to address confidentiality-related vulnerabilities~\cite{foreshadow,
        amd-sev-unprotected-io, severed}.
        %
        Based on this history, even if enclaves are available, the organization may be
        unwilling to provision an enclave with sensitive data.

    \item[P4] \textbf{Support multi-tenancy:} 
        %
        For provider services that muliplex
        across multiple organizations, the system should support multiple
        organizations, and with strong isolation between them.

    \item[P5] \textbf{Maintain performance:}
        %
        Ideally, the performance of the system is similar to that of the vanilla
        system.
\end{enumerate}
