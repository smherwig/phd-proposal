\section{Background}
\label{sec:background}

There are two oraganization, which is the user

% trust model
% goal 



\subsection{Problem}

There are two broad flavors of outsourcing.
%
The first is an organization migrating their in-house service to a third-party
deployment; the second is an organization  migrating from in-house services to
those offered by third party cloud service providers.
%
Hybrids exist between these two extremes, such as where a cloud service
provider must communicate with the organization's, in-house, backend services.
%
Common examples are an organization moving their HTTPS, DNS, or email servers
into the cloud, or contracting some other party to run their own
implementations of these services on the organization's behalf.
%
Nowadays, all of these core application-level protocols use public key
cryptopraphy to ensure the integrity, authenticity, and/or confidentiality of
the application's content.


As a concrete example, consider an organization that outsources its web hosting
to a a content delivery network (CDN)\@.
%
A CDN is a massive, global network of \emph{edge servers}, where each edge
server acts as a reverse proxy for the organization: to
handle client requests, edge servers retrieve content from the organization's
\emph{origin servers}, and cache it so they can deliver it locally.
%
CDNs derive much of their utility from the fact that they have servers close to
most clients, thereby providing low-latency responses.
%
Several techniques exist for routing clients to the closest edge server, but
DNS is common, and so the CDN may also handle teh organization's DNS server.
%
CDNs also provide added security benefits, such as absorbing DDoS
traffic, and filtering targeted malicious trafic, such as SQL injection and
cross-site scripting attacks~\cite{securing-cdns}.
%
Virtually all of the most popular websites (and a very long tail of unpopular
websites) use one or more CDNs to help reliably host their content.


With these use-cases in mind, let's enumerate the security and privacy
implications of outsourcing.


% XXX: try to keep each implication to one meaty (3-5 sentence) paragraph.

\parhead{Impersonation}
%
In order for the provider to fulfill the service, the provider must
have access to the application's private keys.
%
Indeed, as the web moves towards HTTPS-everywhere~\cite{felt-2017-https},
organizations increasingly rely on the CDN provider to store their HTTPS
certificate and the corresponding secret keys~\cite{key-sharing,
when-https-meets-cdn}, so that they can accept TLS connections while
maintaining low latency to clients.
%
Even with application like DNS, where an organiation would traditionally sign
their DNS records offline, the organization may choose to outsource the keys
and management to the provider.
%
% TODO: cite cloudflare blog
With the organization's private DNS keys online, the provider can offer
additional services, such as greater flexibility to return geographically-based
responses, as well as zone enumeration defenses.
%
This has significant implications on the trust model of the PKI and the web
writ large: today's providers can arbitrarily impersonate any of their customer
organizations.


\parhead{Breach of Confidentiality}
%
With the provider acting a man-in-the-middle between the
organization and its clients, the provider is privy to the client's private
data, such as passwords, cookies, and personal content, as well as statistics
for client profiling.
%
The provider might further subcontract aspects of the deployment (as with an
email provider using another provider for virus and spam scanning), such that
the total data exposure is not apparant to the client or organization.


\parhead{Latent Trust Assumptions}
%
Assume a hybrid deployment where the organization has migrated its front-end
business applications to a cloud service provider, but that the application's
database remains within the organization’s network. 
%
Suppose further that the the organization has in-house applications that
continue to use the database. 
%
Before migrating to the cloud, the database could trust that any connections to
it were from benign, internal tools; now, the organization must retrofit a
firewall and privilege model onto the database in order to guard against
potentially malacious or devious queries from the cloud front-end.
%Alternatively, policy compliance can rely on fine-grained access-control
%support in the underlying database management system (DBMS).  
%%
%Unfortunately, the extent of the support and the language used to express the
%policies varies across DBMSs.  
%%
%For instance, PostreSQL offers row security policies that restrict, on a
%per-user basis, which rows can be returns by queries, or inserted, updated, or
%deleted by data modification commands.  
%%
%In contrast, MySQL offers plugins for application-level firewalls and 
%data masking and de-indentification.
%%
%Often, to achieve a fine-granularity, these features must be encompassed
%through virtual tables.
%%
%For many NoSQL databases, such as MongoDB, the underlying desing assumption is
%that the database operates on localhost, with trusted input, and the
%database itself offers does not offer a permission system.


\parhead{Correctness}
% concern 4: correcntess: e.g., DNSSEC providers not actually providing DNSSEC.


\parhead{Legal and Regulatory Restrictions}
% concern 5: legal, policy restrictions
% talk about akamai not hosting content on thrid parties.


\subsection{Trust Model}

\begin{table}[t]
%\resizebox{.5\textwidth}{!}{
\small
\centering
\rowcolors{2}{gray!15}{white}
    \begin{tabular}{@{}lcccl@{}}
        & \textbf{Enclave}& \textbf{Provider} & \textbf{Software} & \textbf{Deployment} \\
        \hline
        1 & \cmark          & \cmark            & \cmark          & standard    \\
        2 &                 & \cmark            & \cmark          & standard    \\
        3 & \cmark          & \cmark            &                 & nuanced     \\
        4 &                 & \cmark            &                 & nuanced     \\
        5 & \cmark          &                   & \cmark          & Conclaves   \\
        6 &                 &                   & \cmark          & Gemini      \\
        7 & \cmark          &                   &                 & out of luck \\ % sandbox in enclave
        8 &                 &                   &                 & out of luck \\ % sandbox in partition
\end{tabular}
%}
\caption{The deployment strategy, depending on whether the provider offers
    trusted enclaves, whether the organization trusts the service provider, and
    whether the organization trusts the provider's application software.
    }
\label{tab:trust-models}
\end{table}

We describe an idealized trust model from the perspective of an
organization that uses a cloud service provider that requires the organization's
sensitive data.
%
For simplicity, we assume that the provider and the cloud machine owner are the
same party, and that the provider administers all software on the machine,
including any system software, such as the operating system, as well as the
application.
%
An untrusted provider may deviate arbitrarily from the stated service, but
remains subject to standard cryptographic assumptions.
%
Likewise, untrusted software may deviate abitrarily from its stated purpose
and, in particular, may actively try to leak sensitive data.
%
We assume, however, that the software is bug-free, and thus the model omits
external actors, such as a remote attacker.
%
% TODO: what about components that proprieatry to the organization?
The application may be composed entirely of open-source components, private
components that are proprietary to the provider, or some combination thereof.
%
If trusted boot is available  (either via a TPM, or in the form of secure
enclave launch), we assume that the organization and provider can agree to run
a specific build of system software, and that such software is trusted and
bug-free.


Table~\ref{tab:trust-models}, shows the different possible trust models based
on whether the provider offers hardware enclaves, whether the organization
trusts the provider, and whether the organization trusts the provider's
software.


\parhead{Models we ignore}
In models 1 and 2, the organization trusts both
the provider and the software, and thus the application may be run in a
conventional environment.
%
Models 3 and 4 are nuanced, as the organization trusts the provider, but not
the software; the software itself might be proprietary, or simply too large for
the organization to properly vet.
%
We choose not to focus on these models as they either disobey an assumption on
the transitivity of trust (that is, a trusted provider deploying a non-trusted,
proprietary, application), or speak to the case of running buggy software,
the latter of which is both outside of our trust model and mitigated with
existing sandbox techniques.


% In the figure, use colors to show the models you treat in the thesis
\parhead{Models we focus on}
In model 5, the organization trusts the software, but does not trust the
provider to run it faithfully, and thus uses one of various
approaches~\cite{talos,haven,scone,graphene} to
run an unmodified legacy application in an enclave.
%
In contrast, in model 6, an enclave is not available (or not trusted to be
immune to side-channnel attacks).
%
This is the starting use-case for Gemini.
%
In models 7 and 8, where neither the provider nor the software is trusted, no
general prupose solution exists for an organization to securely use an
unmodified, legacy, application.


%For instance, the application may be composed of several components, some of
%which are trusted, and some are not; we assume that the organization
%trusts the components that interact with the senstive data.

In practice, Gemini's trust model is more nuanced, and the stringent
black-or-white depiction of ``trusted" is too coarse.
%
For instance, it may be the case that most of the application is untrusted or
otherwise proprietary, but the organization trusts a small portion of the
application, such as its cryptographic library.
%
This scenario is between models 6 and models 7 and 8, as the organization can
identify a small trusted computing base within the application.
%
Gemini handles this scenario, with enclave availability and enclave trust
treated as configuration details.


\subsection{Goals}

\begin{enumerate}
    \item \textbf{Support legacy customer applications}
    \item \textbf{Protect private keys}
    \item \textbf{Correctness:} Ensure that the provider is running the actual
        service as expected.
    \item \textbf{Support multi-tenancy}
    \item \textbf{Usefulness:} The majority of the application's execution must
        take place on the provider's machine.

    \item \textbf{Protect session keys and cookies}

    \item \textbf{Performance:}
\end{enumerate}


%    \item \textbf{Safety:} Only code that the principal trusts, running in an
%        environment that the principal trusts, can access data that the
%        principal has pinned or tainted.
%
%    \item \textbf{Liveness:} Migration does not stop the world.
