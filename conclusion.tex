\section{Conclusion}
\label{sec:conclusion}
% restate: 
%   - problem
%   - work you have done to solve this
%   - work you propose
%   - what success means

In this proposal, I discuss the problem of running applications that
assume a monolithic trust setting under conditions where that assumption no
longer holds, as due to a multi-party deployment or input data.
%
I posit that a major difficulty in solving this problem is the lack of support
that operating systems and programming languages offer for expressing
computation over an execution model that involves heterogeneous trust.
%
My approach is to apply operating system designs and fine-grained information
flow control techniques to hoist, or otherwise partition, the execution
environment into trust boundaries, thereby making the details of moving the
execution between trust domains, and restricting information flow among domains, a
concern of the run-time execution environment, rather than the application
developer.
% 
A primary goal of this approach is to keep the changes to the execution environment
transparent to the applications, such that these techniques allow for
\emph{post-hoc} refinements of a legacy application's trust model.
% 
I describe my prior work of enhancing library operating system for Intel SGX
enclaves,and propose codomains, an execution model tha allows an application to
dynamically switch execution to different domains.
%
If successful, I will enable application deployment options that preserve
the privacy of the involved parties, as well as support the development of
applications that securely leverage the collective value of private datasets.
