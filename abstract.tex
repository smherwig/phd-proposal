\begin{abstract}

Organizations shift the deployment of their services to untrusted third
parties, such as cloud infrastructures, content delivery networks, or email
providers, in order to reduce costs, increase availability, or gain
additional protections inherent to these hosting environments.
%
Unfortunately, such a shift poses a security tradeoff: third-parties that
provide or host the application potentially gain access to sensitive
information about the organization or its clients.


In my proposal, I ask whether it is possible to run legacy application binaries
with confidentiality and integrity guarantees that reflect the
organization's trust with respect to the application and its deployment
setting.
%
The constraint of running unmodified, legacy applications implies that the
enforcement of such guarantees is the responsibilty of the applications's
run-time execution environment. 
%
Since the execution environment is transparent to the application, the insight
is that it may be modified, partitioned, and distributed across domains of
varying trustworthiness, so as to reflect the security goals.


In the first part of my proposal, I review my prior work in extending a
library operating system that runs within an Intel SGX secure hardware
enclave, so as to support running a broader set of trusted, legacy,
applications in untrusted environments.
%
In the second part, I propose a new run-time system, \emph{Gemini}, that is
agnostic to the availability of secure hardware and the trustworthiness of
the application.
%
Gemini presents two complementary abstractions: \emph{distributed containers},
where organizations pin sensitive data to domains that they trust, and
\emph{policy monitors}---modules that an organization installs in a trusted
environment to enforce expected behavior of untrusted applications.
%
I present the design of Gemini and propose a set of evaluations for assessing
its correctness and performance.

\end{abstract}
