\begin{abstract}

Organizations shift the deployment of their services to untrusted third
parties, such as cloud infrastructures, content delivery networks, or email
providers, in order to reduce costs, increase availability, or gain
additional protections inherent to these hosting environments.
%
Unfortunately, such a shift poses a security tradeoff: third-parties that
provide or host the application potentially gain access to sensitive
information about the organization or its clients.


In my proposal, I ask whether it is possible to run legacy application binaries
with confidentiality and integrity guarantees that reflect the
organization's trust with respect to the application and its deployment
setting.
%
The constraint of running unmodified, legacy applications implies that the
enforcement of such guarantees is the responsibilty of the applications's
run-time execution environment. 
%
Since the execution environment is transparent to the application, the insight
is that it may be modified, partitioned, and distributed across domains of
varying trustworthiness, so as to reflect the security goals.


In the first part of my proposal, I review my prior work in extending a
library operating system that runs within an Intel SGX secure hardware
enclave, so as to support running a broader set of trusted, legacy,
applications in untrusted environments.
%
In the second part, I discuss my proposed work, an operating system abstraction
that I call \emph{codomains}. 
%    
Codomains maintain the source-level abstraction of a monolithic program, but
allow applications to, dynamically and at run-time, switch exeuction to
different domains---hosts and enclaves---via language-neutral mechanisms.
%
Within a domain, the application may pin data as only accessible to the domain
itself, or to a subset of peer domains.
%
In contrast to my prior work, \emph{codomains} are not strictly dependent on
secure hardware, but may use such hardware, if available.
%
I present the design of \emph{codomains},  and propose a set of evaluations for
demonstrating their use-cases, as well as assessing their semantic
correctness and performance.
%
Although codomains require some application source-level changes, I discuss
application interposition techniques to remove this requirement.
\end{abstract}


