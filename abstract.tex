\begin{abstract}
Content owners shift the deployment of their services to untrusted third parties,
such as cloud infrastructures, content delivery networks, or peer-to-peer
networks, in order to reduce costs, increase availability, or gain
additional protections inherent to these hosting environments.
%
Unfortunately, such a shift poses a security tradeoff: third-parties that
take part in the application's distribution potentially gain access to
sensitive information about the application, its owner, or its users.


Running applications with strong security and privacy guarantees on untrusted
third parties is a large and active area of research; it spans uses of
trusted hardware and functional encryption that seek to continue hosting
applications completely on third parties, to (re-)designing applications and
protocols so as to delegate trust, preserve privacy, or otherwise partition
the application's execution across trust boundaries.
%
Unfortunately, this prior work either requires modifications to the
application, or has severe limits on the types of unmodified applications
supported.


I ask, therefore, whether it is possible to add security and privacy
guarantees to arbitrary, unmodified applications that were not written with
such properties in mind, for the purposes of running these applications on
untrusted hosts.
%
This question casts the problem of a posteriori augmenting an application's
security and privacy as a responsibility of the run-time execution environment.
%
Since the execution environment is transparent to the application, the
insight is that it may be modified, partitioned, and distributed across nodes
of varying trustworthiness, so as to reflect the security goals of the
content owner.


I approach the problem from two perspectives: one in which I assume the
availability of trusted hardware, such as Intel SGX, to provide a trusted
execution environment (TEE), and one in which I do not assume trusted hardware.
%
For the TEE approach, I extend prior work on SGX-based library operating systems
(libOS) by modifying the libOS to support multi-process, shared resource,
applications.
%
Specifically, my work adapts a libOS into a distributed microkernel, where
each kernel service (for instance, a filesystem or shared memory) runs in a
separate SGX enclave and mediates the service's shared resources among the
application's processes.

For the non-trusted hardware approach, which is the bulk of my new, proposed
work, I incorporates ideas from information-flow control and thread migration.
%
Specifically, I seek to pin private data to host machines, specify
access controls on the pinned data, extend access controls as the data taints
the application's memory space and filesystem, and transfer execution to the
data-owning host when an unauthorized host attempts to access such data.
%
I also discuss opportunities for composing the two approaches, as in a
heterogeneous environment of trusted and non-trusted hardware.


I evaluate my work by demonstrating correctness, measuring performance
overheads, and measuring the ``closeness" of the emergent protocols to
proposed alternatives.
%
I focus on the use cases of deploying TLS-enabled web services on content
delivery networks (CDNs) without granting the CDN operator the private TLS
key, as well as running database applications on untrusted third
parties while maintaining the privacy of the database content.
\end{abstract}
