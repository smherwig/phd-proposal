\begin{abstract}
% Set the stage and problem
Content owners shift the deployment of their services to untrusted third parties,
such as cloud infrastructures, content delivery networks, or peer-to-peer
networks, in order to reduce costs, increase availability, or gain
additional protections inherent to these hosting environments.
Unfortunately, such a shift poses a security tradeoff: third-parties that
take part in the application's distribution potentially gain access to
sensitive information about the application, its owner, or its users.


% Assert insufficiency of existing solutions
Some applications may be suitably re-designed to address these concerns by
partitioning the application into trusted and untrusted components, wherein
the content owner hosts the trusted component.  However, re-designs may be
costly, and are only appropriate if they avoid shifting major hosting
responsbilities back to the content owner.  In other cases, there is no
clear modification to the application that achieves the desired security
goals without fundamentally changing the service.


% Broach approach
Rather than modify the application directly, I propose instead to alter the
execution environment in which the application runs, so as to enforce the
content owner's desired security and privacy properties.  For maximum
generality, I propose to alter the execution environment of unmodified
binaries, rather than the runtime for a specific language.  Since the
execution environment is transparent to the application, the insight is that it
can be componentized for, and distributed across, nodes of varying
trustworthiness, so as to reflect the latent trust assumptions of the
application.


% Describe different threads of the approach and why they are spanning
Within this space of distributed execution environments, there are four key
design choices: (1) the level in the execution environment at which to
interpose, (2) the method for componentizing the environment, (3) the
mechanism for communicating between the resultant components, and (4) the types
of nodes on which these components may run.  I propose to
design and implement solutions that span this space, so as to understand
the advantages and disadvantages of each design choice.  I 
explore interposing on the application's machine code, as by
dynamic binary instrumention, as well as interposing on the operating system
services that the application uses, such as the dynamic loader, C runtime, and
kernel subsystems.  For the latter, I consider cases where traditional
operating system services, such as storage and network communications, are
implemented in user-space or in a distributed fashion.  For componentization, I
explore both static partitions of the environment, as well as dynamic
partitions based on information flow control rules and runtime information.
For communication, I consider message passing and thread migration.  For
the nodes running the execution environment, I consider both secure
hardware enclaves as well as traditional hardware.


% Scope use-cases you will consider/evaluate
I evaluate my implementations through the ubiquitous and important use case of
a content provider deploying a web service via a third party.  I consider
three deployment scenarios, each of which entails different assumptions
about the third parties involved, notions of execution environment, and
security and privacy concerns.  Specifically, I evaluate the deployment of
a website (1) over a content delivery network, (2) as a Tor hidden service,
and (3) over a decentralized peer-to-peer network.
\end{abstract}
