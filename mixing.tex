\section{Mixing Gemini and Conclaves}
\label{sec:mixing}

It may be possible to mix Gemini and conclaves for performance gains or
deployment flexibility. 
%
By mixing, I am referring to composition (nesting the two execution
environments), splitting (transfering between the two environments), and
combintations thereof.

\begin{widelist}
\item \textbf{Enclaves(Gemini(app))} 
    %
    A nested execution environment where Gemini runs on top of an enclaved
    operating system.
    %
    This is likely an optimization in that organizaitons may provision private
    data to the environment, rather than rely on Gemini's execution migration.  
    %
    As opposed to just enclaves, this solution has the benefits of constraining
    malicious behavior of untrusted applications via Gemini's integrity
    policies.
    %
    This solution is likely more applicable to AMD-SEV as opposed to Intel SGX
    enclaves.

\item \textbf{Gemini(Enclaves(app))} 
    %
    A nested execution environment where Gemini manages a collection of
    enclaves: Gemini enforces information flow rules amongst the enclaves and
    also enforces integrity policies on the data flows.
    %
    This solution is likely more applicable to an Intel SGX as opposed to
    AMD-SEV enclaves.


\item \textbf{Gemini(app) + Enclaves(app)):}
    %
    A split execution environment where the application migrates between the
    two enviroments (or between compositions of the two environments).
    %
    For instance, under Gemini, a could process locally migrated into an Intel
    SGX enclave.
    %
    This is likely an optimization over a simple composition of the two
    environments.
    

    There is also a picojump-inspired use-case of moving sensitive code to data.
    %
    This use-case assumes that the data is too voluminous to transfer.
    %
    If the data is non-sensitive, no further work is needed; if the data is
    sensitive, then the data machine may require that taint-tracking run within
    the enclave, and apply egress filters on the the return migration.
    %
    This assumes that the enclaved process will not launder taint.
\end{widelist}
